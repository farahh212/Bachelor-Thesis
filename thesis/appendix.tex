\appendix
\renewcommand{\appendixtocname}{Appendix}
\renewcommand{\appendixpagename}{\appendixtocname}
\addappheadtotoc
\setboolean{@twoside}{false}
\setlength\parskip{.1\baselineskip plus .05\baselineskip minus .05\baselineskip}
% Reduce spacing around sections and subsections
\makeatletter
\renewcommand{\@seccntformat}[1]{\csname the#1\endcsname\quad}
% Reduce spacing before/after sections
\@addtoreset{section}{chapter}
\renewcommand{\section}{\@startsection{section}{1}{\z@}%
  {-3.5ex \@plus -1ex \@minus -.2ex}%
  {2.3ex \@plus.2ex}%
  {\normalfont\Large\bfseries}}
\renewcommand{\subsection}{\@startsection{subsection}{2}{\z@}%
  {-3.25ex\@plus -1ex \@minus -.2ex}%
  {1.5ex \@plus .2ex}%
  {\normalfont\large\bfseries}}
\makeatother

\chapter{Nomenclature}
\label{ch:nomenclature}

This section defines the mathematical notation and symbols used throughout this thesis. Symbols are organized by category for clarity.

\section*{Geometric Parameters}

\begin{itemize}
    \item $d$ --- Shaft diameter (minor diameter for splines)
    \item $d_i$ --- Shaft inner diameter (for hollow shafts)
    \item $D$ --- Hub outer diameter (for press fits) or spline major diameter
    \item $L$ --- Engagement length (hub length, key length, or spline length)
    \item $b$ --- Key width or spline width
    \item $h$ --- Key height
    \item $h_{\text{proj}}$ --- Projected spline tooth height
    \item $h_{\text{eff}}$ --- Effective spline tooth height
    \item $z$ --- Number of spline teeth
    \item $r_m$ --- Mean radius (for splines)
    \item $m$ --- Spline module
    \item $Q_A$ --- Hub diameter ratio, $Q_A = d/D$
    \item $Q_I$ --- Shaft inner diameter ratio, $Q_I = d_i/d$ (zero for solid shafts)
\end{itemize}

\section*{Material Properties}

\begin{itemize}
    \item $E$ --- Young's modulus
    \item $\nu$ --- Poisson's ratio
    \item $\sigma_y$ --- Yield strength
    \item $\sigma_{\text{uts}}$ --- Ultimate tensile strength
    \item $\sigma_{\text{zul}}$ --- Allowable stress
    \item $S_F$ --- Safety factor for yield-based limits
    \item $S_B$ --- Safety factor for ultimate-based limits
\end{itemize}

\section*{Stresses and Pressures}

\begin{itemize}
    \item $p$ --- Interface pressure (general)
    \item $p_{\text{allow}}$ --- Allowable interface pressure
    \item $p_{\text{zul}}$ --- Permissible interface pressure (DIN notation)
    \item $p_{\text{req}}$ --- Required interface pressure
    \item $p_{\text{erf}}$ --- Required interface pressure (German: erforderlich)
    \item $p_{\text{allow,key}}$ --- Allowable bearing pressure for keys
    \item $p_{\text{allow,spline}}$ --- Allowable bearing pressure for splines
    \item $p_{\text{allow,eff}}$ --- Effective allowable pressure (minimum of shaft and hub)
    \item $\tau_{\text{allow}}$ --- Allowable shear stress
\end{itemize}

\section*{Torque and Forces}

\begin{itemize}
    \item $M_{\text{req}}$ --- Required torque
    \item $M_{\text{design}}$ --- Design torque (required torque multiplied by safety factor)
    \item $M_t$ --- Torque capacity
    \item $M_{t,\text{press}}$ --- Press-fit torque capacity
    \item $M_{t,\text{key}}$ --- Key torque capacity
    \item $M_{t,\text{spline}}$ --- Spline torque capacity
    \item $T_\tau$ --- Shear-limited torque capacity (keys)
    \item $T_p$ --- Bearing-pressure-limited torque capacity (keys)
    \item $F_u$ --- Tangential force
\end{itemize}

\section*{Friction and Interference}

\begin{itemize}
    \item $\mu$ --- Friction coefficient (Haftbeiwert in DIN~7190)
    \item $U_e$ --- Elastic interference
    \item $U_w$ --- Working interference (after accounting for surface roughness)
    \item $G$ --- Smoothing loss due to surface roughness
\end{itemize}

\section*{Surface Properties}

\begin{itemize}
    \item $R_z$ --- Surface roughness (arithmetic mean of profile heights)
    \item $R_{z,\text{shaft}}$ --- Shaft surface roughness
    \item $R_{z,\text{hub}}$ --- Hub surface roughness
\end{itemize}

\section*{Safety Factors and Coefficients}

\begin{itemize}
    \item $S$ --- General safety factor
    \item $S_R$ --- Torque safety factor
    \item $K$ --- Load sharing factor (for splines)
\end{itemize}

\section*{Other Parameters}

\begin{itemize}
    \item $A_{\text{contact}}$ --- Contact area
    \item $A_{\text{proj}}$ --- Projected area (for splines)
    \item $A_{\text{shear}}$ --- Shear area (for keys)
    \item $A_{\text{bear}}$ --- Bearing area (for keys)
    \item $r$ --- Radius (general)
\end{itemize}



\chapter{Lists}
\addcontentsline{toc}{section}{List of Abbreviations}
\begin{acronym}[\hspace{3cm}]
  \acro{ML}[ML]{Machine Learning}
  \acro{DIN}[DIN]{Deutsches Institut für Normung (German Institute for Standardization)}
  \acro{API}[API]{Application Programming Interface}
  \acro{UI}[UI]{User Interface}
  \acro{AHP}[AHP]{Analytical Hierarchy Process}
  \acro{CAD}[CAD]{Computer-Aided Design}
  \acro{CAE}[CAE]{Computer-Aided Engineering}
\end{acronym}
\clearpage
\listoffigures
\addcontentsline{toc}{section}{List of Figures}
\listoftables
\addcontentsline{toc}{section}{List of Tables}

\chapter{Source Code}
\label{ch:source_code}
This appendix provides key code snippets from the most critical implementation components. The complete source code is available in the \texttt{Bachelor\_Code} directory.

\clearpage

\section{Analytical Selector: Core Functions}
\label{sec:code_analytical}

\subsection{Press-Fit Capacity Calculation}
\label{subsec:code_pressfit}
\FloatBarrier
\begin{tcolorbox}[colback=gray!5, colframe=gray!50!black, title=Press-Fit Capacity Function, breakable=false]
\begin{lstlisting}
def pressfit_capacity(M_req_Nmm: float, d_mm: float, L_mm: float,
                      shaft_type: str, shaft_mat_name: str, 
                      hub_mat_name: str, mu: float, S_R: float,
                      DiI_mm: Optional[float], DaA_mm: Optional[float]) 
                      -> Dict[str, Any]:
    """Calculate press-fit torque capacity with interference check."""
    # Required pressure for design torque
    p_erf = p_required_pressfit(M_req_Nmm, d_mm, L_mm, mu, S_R)
    
    # Allowable pressure from material limits
    p_zul = p_allow_pressfit(shaft_type, materials[shaft_mat_name],
                             materials[hub_mat_name], d_mm, DiI_mm, DaA_mm)
    
    # Torque capacity from allowable pressure
    Mt_from_pzul = (math.pi * mu * p_zul * L_mm * (d_mm ** 2)) / 2.0
    
    # Interference manufacturability check
    intr = pressfit_interference_check(d_mm, p_erf, shaft_type,
                                       shaft_mat_name, hub_mat_name,
                                       DiI_mm, DaA_mm)
    
    return {"Mt_from_pzul": Mt_from_pzul, "interference": intr, ...}
\end{lstlisting}
\end{tcolorbox}

\clearpage

\subsection{Key Capacity Calculation}
\label{subsec:code_key}

\FloatBarrier
\begin{tcolorbox}[colback=gray!5, colframe=gray!50!black, title=Key Capacity Function, breakable=false]
\begin{lstlisting}
def key_capacity(d_mm: float, l_key_mm: float,
                 shaft_mat_name: str, hub_mat_name: str) -> Dict[str, Any]:
    """Calculate key torque capacity (shear and bearing limits)."""
    b_mm, h_mm = key_geometry_from_d(d_mm)  # DIN 6885 lookup
    
    mat_shaft = materials[shaft_mat_name]
    mat_hub = materials[hub_mat_name]
    
    # Effective allowable: minimum of shaft and hub
    tau_allow = float(mat_shaft["tau_allow_key"])
    p_allow_effective = min(float(mat_shaft["p_allow_key"]),
                         float(mat_hub["p_allow_key"]))
    
    # Calculate both failure modes
    A_shear = b_mm * l_key_mm
    A_bear = (h_mm / 2.0) * l_key_mm
    r = 0.5 * d_mm
    
    T_tau = tau_allow * A_shear * r      # Shear capacity
    T_p = p_allow_effective * A_bear * r # Bearing capacity
    
    # Governing failure mode (minimum)
    return {"Mt": min(T_tau, T_p), ...}
\end{lstlisting}
\end{tcolorbox}
\clearpage
\subsection{Spline Capacity Calculation}
\label{subsec:code_spline}

\FloatBarrier
\begin{tcolorbox}[colback=gray!5, colframe=gray!50!black, title=Spline Capacity Function, breakable=false]
\begin{lstlisting}
def spline_capacity(d_mm: float, L_mm: float,
                    shaft_mat_name: str, hub_mat_name: str,
                    major_d_override: Optional[float] = None,
                    tooth_count_override: Optional[int] = None) 
                    -> Dict[str, Any]:
    """Calculate spline torque capacity with load sharing factor."""
    # Geometry lookup (DIN 5480)
    z, h_proj_mm, D_mm, b_table_mm = spline_geometry_from_d_lookup(d_mm)
    
    # Effective allowable pressure
    mat_shaft = materials[shaft_mat_name]
    mat_hub = materials[hub_mat_name]
    p_allow_effective = min(float(mat_shaft["p_allow_spline"]),
                         float(mat_hub["p_allow_spline"]))
    
    # Effective parameters
    r_m = 0.25 * (d_mm + D_mm)      # Mean radius
    h_eff = 0.8 * h_proj_mm          # Effective flank height
    K = 0.75                          # Load sharing factor
    
    # Torque capacity
    Mt = K * L_mm * z * h_eff * r_m * p_allow_effective
    
    return {"Mt": Mt, "z": z, "D_mm": D_mm, ...}
\end{lstlisting}
\end{tcolorbox}
\clearpage
\subsection{Preference Scoring Function}
\label{subsec:code_scoring}
\FloatBarrier
\begin{tcolorbox}[colback=gray!5, colframe=gray!50!black, title=Scoring Function, breakable=false]
\begin{lstlisting}
def score_candidate(conn: str, Mt_cap: float, M_req: float,
                    d_mm: float, L_mm: float, prefs: UserPrefs,
                    DaA_mm: Optional[float] = None) -> float:
    """Compute composite score for connection candidate."""
    # 1) Margin reward (capped at 35%)
    margin_raw = max(0.0, (Mt_cap - M_req) / max(M_req, 1e-6))
    margin_useful = min(margin_raw, 0.35) / 0.35
    s_margin = 0.10 * margin_useful
    
    # 2) Overdesign penalty (beyond 35%)
    overkill = max(0.0, margin_raw - 0.35)
    s_overkill = -0.10 * min(overkill, 0.5)
    
    # 3) Preference utility (70% weight)
    prof = CONN_PROFILE[conn]
    pref_util = (
        prefs.ease * prof["assembly/disassembly_ease"] +
        prefs.cost * prof["manufacturing_cost"] +
        # ... all 8 preferences ...
    ) / norm
    s_prefs = 0.70 * pref_util
    
    # 4) Connection-specific penalties
    s_hub_stiffness = ...  # Press fit thin hub penalty
    spline_practicality = ...  # Spline penalty when not needed
    
    return s_margin + s_overkill + s_prefs + s_hub_stiffness + spline_practicality
\end{lstlisting}
\end{tcolorbox}
\clearpage
\subsection{Main Selection Function}
\label{subsec:code_main_selector}
\FloatBarrier
\begin{tcolorbox}[colback=gray!5, colframe=gray!50!black, title=Main Selection Function, breakable=false]
\begin{lstlisting}
def select_shaft_connection(request) -> Dict[str, Any]:
    """Main analytical selector function."""
    M_req = float(request.required_torque)
    S_R = float(request.safety_factor)
    M_design = M_req * S_R
    
    # Compute capacities for all three connection types
    pf = pressfit_capacity(M_req, d, L, ...)
    key = key_capacity(d, L, ...)
    spline = spline_capacity(d, L, ...)
    
    # Feasibility filtering
    feasible = {}
    if pf["Mt_from_pzul"] >= M_design and pf["interference"]["ok"]:
        feasible["press"] = pf["Mt_from_pzul"]
    if key["Mt"] >= M_design:
        feasible["key"] = key["Mt"]
    if spline["Mt"] >= M_design:
        feasible["spline"] = spline["Mt"]
    
    if not feasible:
        return {"recommended_connection": "none", "feasible": False}
    
    # Score all feasible candidates
    scores = {conn: score_candidate(conn, Mt_cap, M_design, ...) 
              for conn, Mt_cap in feasible.items()}
    
    # Select highest-scoring candidate
    best_conn = max(scores.items(), key=lambda x: x[1])[0]
    
    return {"recommended_connection": best_conn, "feasible": True, ...}
\end{lstlisting}
\end{tcolorbox}
\clearpage
\section{Synthetic Dataset Generation}
\label{sec:code_dataset}
\FloatBarrier
\begin{tcolorbox}[colback=gray!5, colframe=gray!50!black, title=Dataset Generation Pipeline, breakable=false]
\begin{lstlisting}
def generate_dataset(n_samples: int = 5000, seed: int = 42) -> pd.DataFrame:
    """Generate synthetic dataset using analytical selector as labeling oracle."""
    rng = np.random.default_rng(seed)
    rows: List[Dict] = []
    
    for _ in range(n_samples):
        # Sample all parameters from DIN-compliant distributions
        req = sample_request(rng)  # Diameter, torque, materials, preferences
        
        # Label using analytical selector
        result = select_shaft_connection(req)
        
        if not result.get("feasible", False):
            continue  # Discard infeasible samples
        
        # Store labeled example
        rows.append({
            "shaft_diameter": req.shaft_diameter,
            "hub_length": req.hub_length,
            "required_torque": req.required_torque,
            "pref_ease": req.user_preferences.ease,
            # ... all features ...
            "label": result["recommended_connection"]
        })
    
    return pd.DataFrame(rows)
\end{lstlisting}
\end{tcolorbox}
\clearpage
\section{Machine Learning Training}
\label{sec:code_ml_training}
\FloatBarrier
\begin{tcolorbox}[colback=gray!5, colframe=gray!50!black, title=Model Training Pipeline, breakable=false]
\begin{lstlisting}
def main():
    # Load and prepare data
    df = pd.read_csv(DATASET_PATH)
    X = df[FEATURE_NUMERIC + CATEGORICAL]
    y = LabelEncoder().fit_transform(df["label"])
    
    X_train, X_test, y_train, y_test = train_test_split(
        X, y, test_size=0.2, stratify=y, random_state=42
    )
    
    # Train multiple models
    models = {
        "Random Forest": RandomForestClassifier(n_estimators=150),
        "XGBoost": XGBClassifier(n_estimators=150),
        "LightGBM": LGBMClassifier(n_estimators=150),
        "CatBoost": CatBoostClassifier(n_estimators=150)
    }
    
    best_model = None
    best_score = -1.0
    
    for name, estimator in models.items():
        clf = Pipeline([
            ("preprocess", preprocessor),  # Standard scaling + one-hot encoding
            ("model", estimator)
        ])
        
        clf.fit(X_train, y_train)
        y_pred = clf.predict(X_test)
        
        # Evaluate using macro F1-score
        f1_macro = f1_score(y_test, y_pred, average="macro")
        
        if f1_macro > best_score:
            best_score = f1_macro
            best_model = clf
    
    # Save best model
    joblib.dump(best_model, MODEL_DIR / "connection_classifier.pkl")
\end{lstlisting}
\end{tcolorbox}
\clearpage
\section{Web Application Backend}
\label{sec:code_backend}
\FloatBarrier
\begin{tcolorbox}[colback=gray!5, colframe=gray!50!black, title=FastAPI Endpoint, breakable=false]
\begin{lstlisting}
@app.post("/select-connection", response_model=ConnectionResult)
async def select_connection(request: ShaftConnectionRequest):
    """Main API endpoint combining analytical and ML predictions."""
    try:
        # Analytical prediction
        result = select_shaft_connection(request)
        
        # ML prediction
        ml_features = _assemble_ml_features(request)
        ml_prediction = predict_connection(ml_features)
        
        # Combine results
        result["ml_recommendation"] = ml_prediction["label"]
        result["ml_probabilities"] = ml_prediction["probs"]
        
        return result
    except Exception as e:
        raise HTTPException(status_code=500, detail=str(e))
\end{lstlisting}
\end{tcolorbox}
\clearpage

\section{Web Application Frontend}
\label{sec:code_frontend}

The React frontend (\texttt{shaft-connection-selector/src/App.js}) provides an interactive user interface for inputting design parameters and visualizing results. The frontend communicates with the FastAPI backend via REST API calls, displays both analytical and ML recommendations, and visualizes torque capacities and selection scores.

Screenshots of the frontend interface are provided in Section~\ref{sec:webapp_demo} of the Results chapter, demonstrating the complete user workflow from parameter input through results visualization.
