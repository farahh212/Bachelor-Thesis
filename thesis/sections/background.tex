\chapter{Background}
\label{ch:background}

This chapter establishes the theoretical and conceptual background needed to understand the methodology and framework developed in this thesis for shaft–hub connection selection. It covers the mechanical principles involved in torque transmission, the types of shaft–hub connections considered in this thesis along with their key properties, and the role of DIN/ISO standards in feasibility assessment~\cite{DIN7190_2017,DIN6885_2021,DIN5480_2006}. Since this thesis begins with physics-based calculations and integrates machine learning (ML), essential ML concepts, evaluation metrics, and the motivation for synthetic dataset generation are also presented.

% ---------------------------------------------------------------------------
\section{Motivation: The Shaft--Hub Connection Selection Problem}
\label{sec:background_motivation}

Shaft–hub connections are essential machine elements that form the mechanical interface between a rotating shaft and a hub, allowing torque and rotational motion to be transmitted reliably without relative slip~\cite{Haggenmueller_2025}. These connections play a critical role in the performance of rotating machinery, including electric motors, gearboxes, pumps, conveyors, wind turbines, and automotive drivetrains. Dependable torque transmission and rotational integrity are essential for safe and efficient operation. A suitable shaft–hub connection must therefore meet several key design requirements:
\begin{itemize}
    \item transmit the required torque with adequate safety against slip and local failure,
    \item avoid excessive stress concentrations and preserve fatigue strength,
    \item maintain alignment and concentricity under dynamic loading,
    \item satisfy manufacturing, assembly, and maintenance constraints.
\end{itemize}

Selecting the correct connection type is crucial because failure can lead to
slippage, fretting, shaft damage, fatigue cracking, or catastrophic system
failure. Despite their widespread use, no publicly available labelled datasets
exist for automated shaft--hub connection selection. Industrial practice
typically relies on engineer experience, handbook charts, and repeated
analytical checks, resulting in a manual and time-consuming design process with
limited decision support.

The complexity arises because:
\begin{itemize}
    \item torque capacity depends on interacting geometric, material, and surface parameters;
    \item three distinct connection types (press fit, key, spline) have fundamentally different load paths;
    \item user/application preferences often override pure mechanical capacity;
    \item nonlinear relationships and discrete standardized geometries make manual comparison difficult;
    \item standards (DIN~7190, DIN~6885, DIN~5480) define rules but do not guide connection \emph{selection}.
\end{itemize}

These observations motivate a hybrid analytical--ML approach: analytical models
provide physics-consistent feasibility and transparent capacity values, while ML
provides rapid probabilistic recommendations learned from analytically labelled
synthetic data, and preference weighting enables application-specific trade-offs.

% ---------------------------------------------------------------------------
\section{Shaft--Hub Connections in Rotating Machinery}
\label{sec:background_shc_types}

Connection types fall into two main categories: \emph{friction closure} and
\emph{form closure}. Common designs considered in this thesis are:
\begin{itemize}
    \item \textbf{Interference (press) fits} (friction closure),
    \item \textbf{Parallel keys} (form closure),
    \item \textbf{Splines} (form closure).
\end{itemize}

Each type exhibits characteristic strengths and limitations. In practice, these trade-offs mean that standards-based feasibility alone rarely determines the final design choice; further analysis is typically needed, especially when multiple connection types are feasible.

% ---------------------------------------------------------------------------
\subsection{Friction Closure: Interference (Press) Fits}
\label{sec:press_fits_background}

Press fits transmit torque through friction generated by radial interference between a shaft and a hub. During assembly, the hub may be heated or the shaft cooled (shrink fit), or the parts may be force-assembled at ambient temperature (press fit). After temperature equalisation and elastic recovery, a radial contact pressure $p$ develops at the interface.

For a cylindrical shaft of diameter $d$ and engagement length $L$, the contact area is
\begin{equation}
A_{\text{contact}} = \pi d L .
\end{equation}

Assuming uniform pressure distribution and a friction coefficient $\mu$, the transmissible torque is
\begin{equation}
M_{t,\text{press}} = \mu \, p \, A_{\text{contact}} \, \frac{d}{2}
= \frac{\pi}{2} \, \mu \, p \, L \, d^{2}.
\end{equation}

For a required torque $M_{\text{req}}$ and a torque safety factor $S_R$, the required interface pressure is
\begin{equation}
p_{\text{req}} =
\frac{2 M_{\text{req}} S_R}{\pi \mu L d^{2}} .
\end{equation}

This formulation preserves dimensional consistency and ensures that the safety factor is applied directly to the torque demand.

\paragraph{Allowable Pressure Considerations}
The allowable contact pressure is limited by both shaft and hub constraints. In this work, the effective allowable pressure is therefore defined as
\begin{equation}
p_{\text{allow}} = \min \left( p_{\text{allow,shaft}}, \; p_{\text{allow,hub}} \right).
\end{equation}

While the shaft limit is governed primarily by material strength, the hub limit additionally depends on geometry. A representative geometric ratio is
\begin{equation}
Q = \frac{d}{D},
\end{equation}
where $D$ is the hub outer diameter. As $Q \rightarrow 1$ (thin hubs), the hub's ability to sustain interface pressure decreases significantly, highlighting the sensitivity of press fits to hub thickness.

Mechanical feasibility requires
\begin{equation}
p_{\text{req}} \le p_{\text{allow}} .
\end{equation}

Press fits are compact, backlash-free, and provide excellent concentricity. However, they are sensitive to surface condition, lubrication, and assembly constraints. Extremely high interferences may be torque-feasible in theory but impractical or damaging in assembly. Thin hubs are particularly critical, as hub compliance can lead to deformation such as bell-mouthing. These considerations motivate the additional manufacturability and stiffness checks introduced later in the analytical model.

% ---------------------------------------------------------------------------
\subsection{Form Closure: Keys and Splines}
\label{sec:keys_splines_background}

Form-closure connections transmit torque through geometric interlocking rather than frictional interface pressure. In this work, parallel keys and splines are considered as representative form-closure solutions.

\paragraph{Keys}
A rectangular parallel key engages matching keyways in the shaft and hub. Torque transmission is governed by two primary failure modes: shear of the key and bearing pressure on the key flanks.

The shear-limited torque capacity is
\begin{equation}
T_{\tau} = \tau_{\text{allow}} \, b \, L \, \frac{d}{2},
\end{equation}
while the bearing-pressure-limited capacity is
\begin{equation}
T_{p} = p_{\text{allow}} \left( \frac{h}{2} \right) L \frac{d}{2},
\end{equation}
where $b$ is the key width, $h$ is the key height, and $L$ is the engagement length.

The transmissible torque is therefore
\begin{equation}
M_{t,\text{key}} = \min \left( T_{\tau}, \, T_{p} \right).
\end{equation}

In practice, the bearing pressure limit often governs, particularly for softer hub materials. To reflect this, the allowable bearing pressure is conservatively taken as the minimum allowable value of the shaft--hub material pair. Keys are inexpensive and easy to assemble and disassemble, but they introduce stress concentrations at the keyway, which can reduce fatigue strength and lead to backlash behavior under reversing loads.

\paragraph{Splines}
Splines distribute torque across multiple teeth, significantly increasing load-carrying capacity and improving fatigue performance. The projected flank area may be expressed as
\begin{equation}
A_{\text{proj}} = z \, b \, h_{\text{proj}},
\qquad
h_{\text{proj}} = \frac{D - d}{2},
\end{equation}
where $z$ is the number of teeth.

Rather than using the full projected geometry directly, the torque capacity in this work is based on an effective flank height and a mean radius to account for non-uniform load sharing:
\begin{equation}
M_{t,\text{spline}} =
K \, L \, z \, h_{\text{eff}} \, r_m \, p_{\text{allow}},
\end{equation}
with
\begin{equation}
r_m = \frac{d + D}{4},
\qquad
h_{\text{eff}} \approx 0.8 \, h_{\text{proj}}.
\end{equation}

Splines provide high torque capacity, excellent fatigue behavior, and may be designed for controlled axial sliding. Their main drawbacks are increased manufacturing complexity, tighter tolerances, and higher cost compared to keys and press fits.

% ---------------------------------------------------------------------------
\section{Relevant Industry Standards}
\label{sec:relevant_standards}

Several industry standards inform the analytical foundations of the shaft--hub connection models used in this thesis. These standards provide established formulas, geometric conventions, and limiting criteria for individual connection types. Where standards do not fully specify selection or practical feasibility, conservative engineering heuristics are applied.

\begin{itemize}
    \item \textbf{DIN 7190 (press fits / interference fits)}~\cite{DIN7190_2017}.  
    This standard forms the basis for modeling friction-closure shaft–hub connections. DIN-style friction coefficients (Haftbeiwerte) for common material pairings and surface conditions are used to estimate torque transmission capability~\cite{GWJ_eAssistant_DIN7190,FVA_InterferenceFits_2017}. Allowable interface pressure is governed by safety-factored material limits for both shaft and hub, with the effective limit taken as the minimum of the two. Geometric diameter ratios are used to reflect hub and hollow-shaft effects. In addition, practical interference considerations---accounting for elastic recovery and surface roughness---are incorporated to avoid mechanically feasible but impractical designs.

    \item \textbf{DIN 6885 (parallel keys)}~\cite{DIN6885_2021}.  
    Keyed joints are modeled using standardized key dimensions selected as a function of shaft diameter, consistent with DIN practice. Torque capacity is evaluated based on shear and bearing pressure limits, with conservative allowable values derived from the shaft–hub material combination. The governing capacity is taken as the smaller of these two limits.

    \item \textbf{DIN 5480 (splines)}~\cite{DIN5480_2006}.  
    Splined connections follow a DIN 5480-inspired approach. For smaller diameters, typical spline geometries are selected from a lookup aligned with standard practice. For larger diameters, a module-based heuristic is used to generate plausible spline geometry parameters. Load sharing is represented through conservative reduction factors applied to the effective flank height and torque capacity.
\end{itemize}

In addition to these standards, engineering practicality adjustments are employed to maintain robust and realistic feasibility assessments, such as conservative load-distribution factors and manufacturability checks. While DIN standards define feasibility boundaries, they do not prescribe how to select between multiple feasible connection types. Consequently, this thesis extends standards-based analysis with a preference-weighted and machine-learning-assisted decision framework to support transparent and application-dependent connection selection.


% ---------------------------------------------------------------------------
\section{Materials and Contact Mechanics}
\label{sec:materials_contact_mechanics}

Material properties govern allowable stresses, elastic deformation, and compatibility in shaft--hub connections. The materials considered in this work include representative structural and alloy steels (e.g., C45, 42CrMo4), stainless steels, cast irons, bronzes, and aluminum alloys. For each material, a set of mechanical properties is defined, including Young’s modulus $E$, Poisson’s ratio $\nu$, yield and ultimate tensile strength proxies, and ductility classification. These properties are used to derive safety-factored allowable stresses that limit torque transmission capacity.

For form-closure connections, allowable shear stresses and permissible bearing pressures are assigned on a material-specific basis. In keyed and splined joints, bearing pressure limits are conservatively governed by the weaker component of the shaft--hub material pair, ensuring that local contact stresses remain within admissible bounds for both mating parts.

For friction-closure connections, contact mechanics are governed by interface pressure and friction. The friction coefficient $\mu$ is sensitive to several factors, including:
\begin{itemize}
    \item material pairing (e.g., steel--steel versus steel--aluminum),
    \item surface condition (dry or oiled),
    \item surface roughness,
    \item assembly method (press fit or shrink fit).
\end{itemize}

Conservative friction coefficient ranges inspired by DIN practice are used to estimate torque transmission capacity while maintaining safety against slip. Rather than assigning a single fixed value, friction coefficients are sampled within bounded intervals corresponding to the selected material pairing and surface condition. This approach preserves physical interpretability while introducing controlled variability during synthetic dataset generation.

In addition, elastic material properties, specifically Young’s modulus and Poisson’s ratio, are used to estimate interference-related deformation effects in press-fit connections. These estimates are employed as plausibility checks that account for elastic recovery and surface roughness losses, allowing mechanically feasible but
practically unrealistic designs to be filtered out. Together, these material and contact mechanics considerations provide a physically consistent basis for analytical feasibility assessment and the subsequent decision-making framework developed in
this thesis.

% ---------------------------------------------------------------------------
\section{Feasibility Considerations}
\label{sec:background_analytical_models}

Before different shaft--hub connection types can be meaningfully compared, basic
mechanical feasibility must be ensured. In general, feasibility is established by verifying
that a connection can transmit the required torque with an appropriate safety margin,
while remaining within admissible material and geometric limits.

Beyond strength-related limits, practical and geometric considerations are commonly
required to avoid non-physical or unrealistically difficult designs (e.g., invalid diameter
relationships or configurations that are impractical to assemble). In this thesis, these
feasibility considerations serve as a prerequisite to the subsequent preference-based
evaluation and selection, while the detailed formulation and implementation of the
feasibility checks are presented in Chapter~\ref{ch:methodology}.

% ---------------------------------------------------------------------------
\section{Preference-Based Engineering Trade-Offs}

Engineering decisions often involve trade-offs beyond torque capacity alone.
Accordingly, this work considers multiple qualitative and semi-quantitative preference
dimensions, including:
\begin{itemize}
    \item assembly and disassembly ease,
    \item suitability for axial movement,
    \item cost sensitivity,
    \item bidirectional torque capability,
    \item vibration resistance,
    \item high-speed suitability,
    \item maintenance effort and accessibility,
    \item durability and fatigue-related considerations.
\end{itemize}

Different shaft--hub connection types exhibit characteristic strengths and weaknesses
across these dimensions. Incorporating user-defined preference weighting enables
selection decisions that reflect application-specific priorities rather than relying solely
on mechanical capacity. The detailed formulation of the preference-based evaluation is
introduced in Chapter~\ref{ch:methodology}.

% ---------------------------------------------------------------------------
\section{Synthetic Data in Engineering Design}

Because no labelled datasets exist for shaft--hub connection selection, this thesis
relies on synthetically generated data. Each synthetic sample represents a plausible
engineering design scenario, characterized by:
\begin{itemize}
    \item geometric parameters (e.g., shaft diameter, engagement length, hub diameter,
    solid or hollow shafts),
    \item material combinations and surface conditions,
    \item required torque and safety factor,
    \item user-defined preference weights.
\end{itemize}

A combination of analytical and preference-based steps is used to establish
mechanical feasibility and to identify suitable connection types, resulting in a
physically grounded dataset~\cite{Picard_2023}. This approach enables supervised machine learning
while avoiding the cost and complexity of large-scale experimental data acquisition.
The data generation procedure is detailed in Chapter~\ref{ch:methodology}.


% ---------------------------------------------------------------------------
\section{Machine Learning Concepts for Hybrid Prediction}
\label{sec:background_ml}

\subsection{Supervised Classification}
\label{subsec:background_supervised}

Machine learning provides a complementary decision layer to the analytical
engineering approach. In a supervised classification setting, a model learns a
mapping from input features to a recommended shaft--hub connection type. In this
work, the input features describe the mechanical design context, including geometric
parameters, required torque and safety factor, shaft type, shaft material and surface
condition, as well as user-defined preference weights.

\subsection{Tree-Based Models and Gradient Boosting}
\label{subsec:background_boosting}

Tree-based models are well suited to this problem because they can represent
nonlinear decision boundaries that commonly arise from mechanical feasibility
constraints and can naturally handle a mixture of numerical and categorical inputs.
Accordingly, this thesis considers Random Forest classifiers~\cite{Breiman_2001} as well as several
gradient-boosted tree frameworks, including:
\begin{itemize}
    \item \textbf{XGBoost}: gradient boosting with regularization and second-order
    optimization,
    \item \textbf{LightGBM}: histogram-based gradient boosting optimized for efficient
    training,
    \item \textbf{CatBoost}: gradient boosting with robust handling of categorical
    variables and reduced overfitting tendencies.
\end{itemize}

\subsection{Ensemble Learning}
\label{subsec:background_ensemble}

In addition to individual classifiers, ensemble learning can improve robustness and
generalization. A \emph{soft-voting ensemble} combines multiple base classifiers by
averaging their predicted class probabilities and selecting the class with the highest
average probability. By aggregating partially uncorrelated models, such ensembles can
reduce variance and improve prediction stability.

\subsection{Evaluation Metrics and Model Selection}
\label{subsec:background_metrics}

Classification performance is commonly assessed using metrics that capture both
overall correctness and class-specific behavior. In multi-class settings, it is useful to
report accuracy alongside class-balanced measures such as macro-averaged precision,
macro-averaged recall, and macro-averaged F1-score. In this thesis, model comparison
and selection are based primarily on the macro-averaged F1-score to avoid favoring a
single dominant class, while accuracy is reported as a complementary summary metric.
A confusion matrix is additionally used to visualize systematic confusions between the
three connection types (press fit, key, spline) and to support qualitative error analysis.


% ---------------------------------------------------------------------------
\section{Summary}
\label{sec:background_summary}

This chapter established the mechanical, analytical, and machine-learning
background required to understand the shaft--hub connection selection problem.
Fundamental connection types, relevant DIN standards, material and contact
mechanics considerations, and preference-based engineering trade-offs were
introduced to motivate the need for a structured selection framework. The
chapter also outlined the role of synthetic data and supervised machine learning
as a means of scaling analytical decision logic in the absence of labelled
industrial datasets.

Building on this foundation, the following chapter presents the methodology
used to implement a hybrid analytical--machine learning selector, including
physics-based feasibility assessment, preference-weighted analytical ranking,
synthetic dataset generation, classifier training and evaluation, and system
integration.
