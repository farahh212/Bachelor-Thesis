\chapter{Introduction}

Shaft–hub connections are fundamental components in mechanical engineering. A rotating
shaft transmits torque to an attached component through this interface, and the choice
of connection type significantly affects the integrity and performance of the entire system.

An overly conservative selection can introduce unnecessary cost, manufacturing
complexity, and maintenance burden, while an under-designed connection that does
not satisfy the mechanical transmission requirements can compromise the safety of
the entire system, potentially resulting in slippage, deformation, or catastrophic
failure. Therefore, it is essential to ensure the connection is strong enough to
withstand torque requirements while considering economic constraints and cost
efficiency.

This gives rise to the topic of optimal shaft--hub connection selection,
which is traditionally done using analytical calculations and expert
judgement. However, this process can be time-consuming and impractical
in startups that may not have access to specialized consultation and
expert knowledge.

Recent research has demonstrated the potential of machine-learning models to
support decision-making across a wide variety of engineering tasks~\cite{Breiman_2001}.
However, a significant gap exists: no publicly available dataset labels shaft–hub
connections. This makes the process of training an ML model directly infeasible.

The aim of this thesis is to develop an intelligent tool for assigning a
shaft--hub connection type to a given set of mechanical and
preference-based inputs. This is achieved by designing a scoring system
derived from analytical equations and user preference weighting. This is
then scaled up to generate a synthetic dataset using inputs randomized
based on DIN standards and the scoring system as a label generator. The
resulting dataset is used to train a machine-learning model that
automates the prediction task. This produces a unified tool that blends
analytical feasibility logic, user preferences, and an ML model.

\section{Research Objectives}

To achieve the aim of this thesis, the following research objectives are defined:

\begin{itemize}
    \item Develop a scoring system pipeline capable of assigning a
    shaft--hub connection label based on input parameters and
    preference-driven criteria.

    \item Scale this pipeline to generate a large and diverse synthetic
    dataset using input parameters randomized in accordance with DIN
    standards.

    \item Train and evaluate a machine-learning classification model
    using the synthetic dataset.

    \item Integrate the trained model into a web-based application to
    provide automated shaft--hub connection recommendations.

    \item Present analytical torque capacities and ML confidence scores
    to improve transparency and interpretability of the selection
    process.
\end{itemize}   

\section{Contributions}

This thesis makes the following key contributions:

\begin{itemize}
    \item An analytical scoring system that evaluates shaft--hub
    connections using torque-based feasibility rules and
    preference-weighted criteria.

    \item A large, labeled synthetic dataset for shaft--hub connection
    selection, generated through a DIN-compliant automated pipeline and
    including press fits, keyed fits, and splined fits.

    \item A trained and evaluated machine-learning classifier capable of
    predicting the most suitable shaft--hub connection type based on the
    generated dataset.

    \item A web-based interface, developed using FastAPI and React, that
    provides real-time connection recommendations to users.

    \item A hybrid intelligent framework combining analytical mechanics
    with machine-learning inference, providing torque capacity outputs
    and softmax-based confidence scores for each prediction.
\end{itemize}
 

\section{Structure of the Thesis}

This thesis is organized as follows. Chapter~1 introduces the motivation, 
research gap, aim, and objectives of the work. Chapter~2 provides the 
necessary background on shaft–hub connections, analytical torque transmission 
models, relevant machine-learning concepts, and a summary of related work.

Chapter~3 describes the methodology, including the development of the 
analytical scoring system, the synthetic dataset generation pipeline, and the 
training and evaluation of the machine-learning model. Chapter~4 reports the 
results, covering model performance, analysis of the generated dataset, and 
interpretation of prediction behaviour. Chapter~5 discusses the findings and
their implications. Finally, Chapter~6 concludes the thesis and outlines 
potential directions for future research.

