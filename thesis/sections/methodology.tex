\chapter{Methodology}
\label{ch:methodology}

This chapter presents the methodology used to develop the hybrid
analytical--machine learning framework for shaft--hub connection selection.
The approach addresses the research gap established in Chapter~\ref{ch:background}:
because no publicly available labelled dataset exists for this selection task,
the methodology constructs one by first building a physics-based analytical
selector, then using it as an automated labeling oracle. A supervised classifier
trained on the resulting synthetic data is subsequently integrated alongside
the analytical engine into a deployable decision-support tool.

The methodology proceeds through four stages, each building on the previous:

\begin{enumerate}
    \item \textbf{Analytical selector development.} Torque capacities for press
    fits, keys, and splines are computed using DIN-based equations. Candidates
    that fail to meet the factored design torque are rejected; press fits
    undergo an additional manufacturability check to exclude impractical
    interference values. Feasible candidates are ranked using a preference-weighted
    scoring function that combines capacity margin, application-specific
    performance profiles, and user-supplied weights.

    \item \textbf{Synthetic dataset generation.} Geometry, torque demand,
    materials, surface conditions, and preference weights are sampled within
    realistic, DIN-consistent ranges. Each sample is passed through the
    analytical selector, and the resulting recommendation becomes the class
    label. Infeasible samples are discarded to avoid an ambiguous ``none'' class.

    \item \textbf{Machine learning training.} Multiple tree-based classifiers
    are trained on the synthetic dataset. Models are compared using macro
    F1-score to ensure balanced performance across all connection classes, and
    the best-performing pipeline is persisted with full preprocessing metadata.

    \item \textbf{Deployment.} The analytical selector and trained classifier
    are integrated into a FastAPI backend. A React-based frontend collects user
    inputs and displays both the analytical recommendation (with capacities,
    feasibility status, and scores) and the ML prediction (with class
    probabilities), enabling side-by-side comparison and transparent
    decision-making.
\end{enumerate}

The remainder of this chapter details each stage: Section~\ref{sec:materials_database}
describes the material database; Sections~\ref{sec:analytical_selector}
and~\ref{sec:feasibility_scoring} cover the analytical capacity models and
scoring logic; Section~\ref{sec:synthetic_generation} explains the dataset
generation process; Section~\ref{sec:ml_training} presents the ML training
pipeline; and Section~\ref{sec:deployment} describes the backend and frontend
integration.

% ---------------------------------------------------------------------------
\section{Material Database and Engineering Constants}
\label{sec:materials_database}

Before capacity calculations can proceed, the system requires access to material
properties and application-specific allowables. A curated material database
stores the following properties for each entry:
\begin{itemize}
    \item elastic constants: Young's modulus $E$ and Poisson's ratio $\nu$,
    \item strength values: yield strength $\sigma_y$ and ultimate tensile
          strength $\sigma_{uts}$,
    \item a ductility flag distinguishing ductile from brittle behavior,
    \item safety modifiers $S_F$ (for yield-based limits) and $S_B$ (for
          ultimate-based limits),
    \item a material category (steel, cast iron, bronze, or aluminum) used for
          friction coefficient lookup.
\end{itemize}

The database covers typical engineering materials: structural and alloy steels
(S235, C45, 42CrMo4, E360, 16MnCr5), stainless steel (304), cast irons (GG25,
GGG40), bronze (CuSn8), and aluminum alloys (6061, 7075). Each material also
stores three application allowables:
\begin{align}
    \tau_{\text{allow,key}}    &\quad \text{(key shear allowable)}, \nonumber \\
    p_{\text{allow,key}}       &\quad \text{(keyway bearing allowable)}, \nonumber \\
    p_{\text{allow,spline}}    &\quad \text{(spline flank bearing allowable)}.
\end{align}

For connections involving two different materials (e.g., a steel shaft in a
bronze hub), the effective allowable is taken as the minimum of both components:
\[
p_{\text{allow,eff}} = \min\bigl(p_{\text{allow,shaft}},\, p_{\text{allow,hub}}\bigr).
\]
This conservative rule ensures that neither component exceeds its local stress
limits.

% ---------------------------------------------------------------------------
\section{Analytical Selector: Capacity Computations}
\label{sec:analytical_selector}

The analytical selector evaluates three connection types---press fits, keyed
joints, and splines---by computing their torque capacities from first principles.
Each capacity model draws on the material database and standardized geometry
tables to produce physically meaningful results.

\subsection{Input Validation}
\label{subsec:input_validation}

Every request undergoes validation before capacity calculations begin:
\begin{itemize}
    \item \textbf{Material availability:} both shaft and hub materials must
          exist in the database.
    \item \textbf{Shaft type:} restricted to \texttt{solid} or \texttt{hollow};
          hollow shafts require an inner diameter $d_i$ satisfying $0 < d_i < d$.
    \item \textbf{Required torque:} a mandatory positive value $M_{\text{req}}$
          (in Nmm).
    \item \textbf{Hub geometry:} outer diameter $D > d$ (defaulting to $2d$ if
          unspecified); engagement length $L$ (defaulting to $1.5d$).
\end{itemize}
Invalid inputs raise descriptive exceptions, preventing nonsensical
configurations from propagating through the system or corrupting the synthetic
dataset.

\subsection{Press-Fit Capacity (Friction Closure)}
\label{subsec:pressfit_capacity}

Press fits transmit torque through friction generated by interface pressure
arising from elastic interference. The capacity calculation proceeds in two
stages: determining the allowable interface pressure, then computing the
resulting torque capacity.

\paragraph{Friction coefficient selection.}
The friction factor $\mu$ (termed \emph{Haftbeiwert} in DIN~7190-1) represents
the achievable tangential traction at the interface~\cite{DIN7190_2017,GWJ_eAssistant_DIN7190}.
The system stores conservative ranges keyed by material-category pairs and
surface condition (\texttt{dry} or \texttt{oiled}). For example, steel--steel
dry fits use $\mu \in [0.14, 0.20]$, while steel--cast iron oiled fits use
$\mu = 0.10$. Given a request, $\mu$ is sampled uniformly from the applicable
range:
\[
\mu \sim \mathcal{U}\bigl(\mu_{\min},\, \mu_{\max}\bigr),
\]
introducing controlled variability that later enriches the synthetic dataset.
A user override is supported but clamped to $[0.05, 0.25]$ to prevent
unrealistic friction assumptions.

\paragraph{Allowable pressure.}
The permissible interface pressure $p_{\text{zul}}$ depends on material
strength, ductility, and geometric ratios. Defining
\[
Q_A = \frac{d}{D}, \qquad Q_I = \frac{d_i}{d} \quad (\text{zero for solid shafts}),
\]
the allowable stress for each component is:
\[
\sigma_{\text{zul}} =
\begin{cases}
\sigma_y / S_F, & \text{if ductile}, \\
\sigma_{uts} / S_B, & \text{if brittle}.
\end{cases}
\]
The hub and shaft pressure limits are then:
\begin{align}
p_{\text{hub}} &= \frac{1 - Q_A^2}{\sqrt{3}}\,\sigma_{\text{zul,hub}}, \\
p_{\text{shaft}} &= \frac{2}{\sqrt{3}}\,\sigma_{\text{zul,shaft}}\,(1 - Q_I^2),
\end{align}
with the governing limit $p_{\text{zul}} = \min(p_{\text{hub}},\, p_{\text{shaft}})$.

\paragraph{Torque capacity.}
Using the allowable pressure, the press-fit torque capacity is:
\[
M_{t,\text{press}} = \frac{\pi}{2}\,\mu\,p_{\text{zul}}\,L\,d^2.
\]

\paragraph{Manufacturability filter.}
A press fit can be torque-feasible but impractical if it requires excessive
interference. The selector therefore evaluates an interference plausibility
check. First, the required interface pressure for a given torque with safety
factor $S_R$ is:
\[
p_{\text{req}} = \frac{2\,M_{\text{req}}\,S_R}{\pi\,\mu\,d^2\,L}.
\]
The elastic interference $U_e$ is computed from combined compliance:
\[
U_e = p_{\text{req}}\,d \left[
\frac{1+\nu_I}{E_I(1-Q_I^2)} + \frac{1+\nu_A}{E_A(1-Q_A^2)}
\right],
\]
where subscripts $I$ and $A$ denote shaft and hub respectively. Surface
roughness reduces effective interference through a smoothing loss:
\[
G = 0.4\,\frac{Rz_{\text{shaft}} + Rz_{\text{hub}}}{1000},
\]
with roughness values in $\mu$m and $G$ in mm. The working interference is
$U_w = U_e - G$, subject to the limits:
\[
U_w \le
\begin{cases}
0.02~\text{mm}, & d \le 50~\text{mm}, \\
0.05~\text{mm}, & d > 50~\text{mm}.
\end{cases}
\]
If $U_w \le 0$ (roughness consumes interference) or $U_w$ exceeds the limit,
the press-fit candidate is rejected regardless of its torque capacity.

\subsection{Key Capacity (Form Closure)}
\label{subsec:key_capacity}

Keys transmit torque through shear and bearing at the key--keyway interface.
Geometry is determined from a standardized lookup table mapping shaft diameter
$d$ to key width $b$ and height $h$ per DIN~6885~\cite{DIN6885_2021}. Two failure modes govern
capacity:
\begin{align}
T_\tau &= \tau_{\text{allow}}\,b\,L\,\frac{d}{2} \quad \text{(shear)}, \\
T_p &= p_{\text{allow,eff}}\,\frac{h}{2}\,L\,\frac{d}{2} \quad \text{(bearing)},
\end{align}
with the key torque capacity:
\[
M_{t,\text{key}} = \min(T_\tau,\, T_p).
\]
The shear allowable $\tau_{\text{allow}}$ comes from the shaft material (the
key is typically made from the same or weaker stock), while
$p_{\text{allow,eff}}$ uses the weaker of shaft/hub bearing allowables.

\subsection{Spline Capacity (Form Closure)}
\label{subsec:spline_capacity}

Spline geometry is determined from a lookup table for diameters up to 112~mm,
providing major diameter $D$, tooth count $z$, and width $B$. Beyond this range,
a DIN~5480-like heuristic~\cite{DIN5480_2006} selects a module $m$ from a standard series,
estimates $z \approx d/m$, and computes the major diameter from a projected
tooth height. User overrides for $D$ and $z$ are supported and validated.

Capacity is computed via an effective flank model:
\[
r_m = \frac{d + D}{4}, \qquad h_{\text{eff}} = 0.8\,h_{\text{proj}},
\]
where $h_{\text{proj}} = 0.5(D - d)$. The spline torque capacity is:
\[
M_{t,\text{spline}} = K\,L\,z\,h_{\text{eff}}\,r_m\,p_{\text{allow,eff}},
\]
with $K = 0.75$ representing load-sharing losses and practical non-uniformities.

% ---------------------------------------------------------------------------
\section{Feasibility Filtering and Preference-Based Scoring}
\label{sec:feasibility_scoring}

With capacities computed for all three connection types, the selector applies a
two-stage decision process: first filtering infeasible candidates, then ranking
feasible options using a preference-weighted scoring model.

\subsection{Design Torque and Feasibility}
\label{subsec:design_torque}

The design torque incorporates the user-specified safety factor:
\[
M_{\text{design}} = M_{\text{req}} \cdot S.
\]
A candidate is feasible if its capacity meets or exceeds the design torque:
$M_t \ge M_{\text{design}}$. Press fits additionally require that the
interference plausibility check passes. If no candidate is feasible, the
selector returns \texttt{none} with an explicit reason---for example,
``press-fit torque OK but rejected by interference check.''

\subsection{Connection Performance Profiles}
\label{subsec:conn_profiles}

Each connection type is assigned a fixed performance profile across eight
application dimensions:
\begin{itemize}
    \item assembly/disassembly ease,
    \item axial movement suitability,
    \item manufacturing cost,
    \item bidirectional torque capability,
    \item vibration resistance,
    \item high-speed suitability,
    \item maintenance ease,
    \item durability/fatigue life.
\end{itemize}

For instance, press fits score high on vibration resistance (0.85) and
high-speed suitability (0.90) but low on assembly ease (0.20); splines excel
at axial movement (0.95) and bidirectional torque (0.90) but are expensive to
manufacture (0.20). These profiles encode domain expertise and remain constant
across all requests.

\subsection{Scoring Function}
\label{subsec:scoring_function}

Users specify preference weights for each dimension (0.0--1.0 in 0.1 increments).
The scoring function combines several terms:

\paragraph{Margin reward.}
A diminishing reward for capacity surplus, capped at 35\% margin:
\[
s_{\text{margin}} = w_{\text{margin}} \cdot \min\left(1,\, \frac{M_t - M_{\text{design}}}{0.35\,M_{\text{design}}}\right).
\]

\paragraph{Overdesign penalty.}
A bounded penalty for excessive capacity beyond the useful margin:
\[
s_{\text{overkill}} = -w_{\text{overkill}} \cdot \min\left(0.5,\, \frac{M_t - M_{\text{design}}}{M_{\text{design}}} - 0.35\right)^+.
\]

\paragraph{Preference utility.}
User weights are normalized and combined with the connection profile:
\[
s_{\text{prefs}} = w_{\text{prefs}} \cdot \frac{\sum_i u_i \cdot p_i}{\sum_i u_i},
\]
where $u_i$ are user weights and $p_i$ are profile scores.

\paragraph{Connection-specific penalties.}
Press fits receive a hub stiffness penalty when $Q_A = d/D$ exceeds 0.5
(thin-walled hubs). Splines receive a practicality penalty when their key
advantages (movement, bidirectional, durability) are not valued by the user.

The feasible candidate with the highest composite score becomes the analytical
recommendation. All intermediate values (capacities, scores, interference
diagnostics) are retained for transparency.

% ---------------------------------------------------------------------------
\section{Synthetic Dataset Generation}
\label{sec:synthetic_generation}

Because no labeled dataset exists for shaft--hub connection selection, the
analytical selector serves as an automated labeling oracle. A synthetic dataset
is generated by sampling realistic input configurations and recording the
analytical recommendation as the ground-truth label.

\subsection{Geometry and Condition Sampling}
\label{subsec:geometry_sampling}

Each sample is drawn from distributions designed to reflect realistic
engineering practice:
\begin{itemize}
    \item \textbf{Diameter:} sampled from a discrete DIN-like progression
          (6--230~mm), with 70\% probability mass concentrated in common ranges
          (20--60~mm), 25\% in mid-ranges (60--120~mm), and 5\% in tails.
    \item \textbf{Hub length:} proportional to diameter; bending-dominated cases
          use $L \approx 0.9d$--$1.3d$, while non-bending cases use
          $L \approx 0.4d$--$0.8d$.
    \item \textbf{Shaft type:} 80\% solid, 20\% hollow; hollow shafts sample
          $d_i \in [0.3d,\, 0.6d]$.
    \item \textbf{Hub outer diameter:} sampled in $D \in [1.8d,\, 2.6d]$ with
          slight increases for bending cases.
    \item \textbf{Surface condition:} \texttt{dry} or \texttt{oiled} with equal
          probability; 15\% of samples include a friction coefficient override.
    \item \textbf{Material:} sampled uniformly from the material database; shaft
          and hub use the same material to maintain internal coherence.
\end{itemize}

\subsection{Torque and Safety Factor Sampling}
\label{subsec:torque_sf_sampling}

Torque is sampled relative to a diameter-dependent reference based on the
polar section modulus:
\[
M_{\text{ref}} = 0.05 \cdot c \cdot \frac{\pi\,d^3}{16} \cdot f_{\text{taper}},
\]
where $c$ is a torque coefficient and $f_{\text{taper}}$ reduces demand for
larger diameters ($f_{\text{taper}} = 0.9$ for $d > 40$~mm, 0.8 for $d > 70$~mm).
A multiplicative factor in $[0.3, 1.4]$ generates cases ranging from
conservative to demanding.

Safety factors are sampled around a baseline of 1.5 with adjustments for:
\begin{itemize}
    \item bending present: $+0.10$,
    \item dry surface condition: $+0.05$,
    \item friction override: $-0.05$,
    \item torque factor exceeding reference: $+0.20 \cdot (\text{factor} - 1)$,
    \item durability preference: $+0.20 \cdot (p_{\text{dur}} - 0.5)$,
    \item cost preference: $-0.15 \cdot (p_{\text{cost}} - 0.5)$.
\end{itemize}
Results are clamped to $[1.0, 2.0]$ and rounded to one decimal place.

\subsection{Preference Weight Sampling}
\label{subsec:pref_sampling}

Each of the eight preference weights is sampled independently as a discrete
value from $\{0.0, 0.1, \ldots, 1.0\}$, matching the frontend slider resolution.

\subsection{Label Generation}
\label{subsec:label_generation}

For each sampled configuration, the analytical selector is invoked. Samples
yielding \texttt{none} (infeasible) are discarded by default to avoid
introducing an ambiguous class. The resulting dataset contains approximately
5{,}000 rows with columns for all geometry, torque, safety factor, surface
condition, preferences, and the analytical label.

% ---------------------------------------------------------------------------
\section{Machine Learning Training and Model Selection}
\label{sec:ml_training}

The synthetic dataset trains supervised classifiers to approximate the
analytical selector. The ML component provides rapid probabilistic predictions
and confidence estimates, complementing the slower but fully transparent
analytical path.

\subsection{Feature Engineering}
\label{subsec:feature_engineering}

The feature set comprises:
\begin{itemize}
    \item \textbf{15 numerical features:} shaft diameter, hub length, bending
          flag (0/1), safety factor, hub outer diameter, shaft inner diameter
          (0 for solid shafts), required torque, and eight preference weights.
    \item \textbf{3 categorical features:} shaft type, shaft material, and
          surface condition.
\end{itemize}

A column-wise preprocessor applies standard scaling to numerical features and
one-hot encoding (with unknown-category handling) to categorical features.
Target labels (\texttt{press}, \texttt{key}, \texttt{spline}) are integer-encoded
with the mapping preserved for inference.

\subsection{Model Selection Rationale}
\label{subsec:model_rationale}

Tree-based classifiers are selected for several reasons. First, the feature
space is heterogeneous, mixing continuous variables (diameter, torque) with
discrete categorical variables (material, shaft type). Tree-based methods
naturally handle mixed data types without requiring extensive feature
engineering. Second, mechanical feasibility regions often exhibit nonlinear
decision boundaries---for example, a small diameter change can shift the optimal
connection type, and preference weight interactions create complex trade-offs.
Tree ensembles capture these interactions through hierarchical splits. Third,
tree-based models provide feature importance scores that aid interpretability,
allowing identification of which design parameters most influence the selection.

Neural networks were not considered because the dataset size (approximately
5{,}000 samples) is relatively small, and tree-based methods typically
outperform neural networks on tabular data of this scale~\cite{grinsztajn2022tree}.
Linear models (logistic regression, SVM) were excluded because they cannot
capture the complex interactions between geometry, materials, and preferences
that govern connection selection.

\subsection{Model Candidates}
\label{subsec:model_candidates}

Four gradient-boosted and ensemble tree models are evaluated:
\begin{itemize}
    \item \textbf{Random Forest:} bagged decision trees providing robust
          predictions through variance reduction.
    \item \textbf{XGBoost:} gradient boosting with regularization, known for
          strong performance on structured data.
    \item \textbf{LightGBM:} gradient boosting optimized for speed and memory
          efficiency using leaf-wise growth.
    \item \textbf{CatBoost:} gradient boosting with built-in categorical
          handling and reduced overfitting through ordered boosting.
\end{itemize}
All models use 150 estimators and are wrapped in identical preprocessing pipelines
to ensure fair comparison. A soft-voting ensemble averaging predicted class
probabilities across all four base estimators is also evaluated, providing a
simple method to stabilize predictions when individual models disagree.

\subsection{Evaluation Strategy and Metric Selection}
\label{subsec:evaluation_metrics}

Data is split 80/20 with stratification by class to preserve the distribution of
connection types in both training and test sets. Multiple metrics are computed:
accuracy, macro-averaged precision, recall, and F1-score, plus the confusion
matrix for per-class analysis.

Macro F1-score is chosen as the primary selection criterion for two reasons.
First, it weights all classes equally by computing the F1-score for each
connection type independently, then averaging. This prevents models from
achieving high overall accuracy by excelling only on the most frequent class
while failing on minority classes. Second, F1-score balances precision and
recall, which is important when all three connection types (press, key, spline)
are equally valid solutions depending on application context---misclassifying a
spline as a key is as problematic as misclassifying a key as a press.

Accuracy alone would be insufficient because it can be misleading in multi-class
problems with potentially imbalanced class distributions. Micro-averaged F1
would be equivalent to accuracy in this setting and suffers from the same
limitation. The confusion matrix provides additional diagnostic information
about which classes are most frequently confused, informing potential
improvements to the feature set or model architecture.

The best-performing pipeline (individual model or ensemble) according to macro
F1-score is selected and persisted for deployment. A detailed comparison of
model performance, including per-class metrics and confusion matrices, is
presented in Chapter~\ref{ch:results}.

\subsection{Model Persistence}
\label{subsec:model_persistence}

The selected pipeline is saved alongside a metadata object containing:
\begin{itemize}
    \item the ordered feature list and numeric/categorical partition,
    \item the selected model name and performance metrics,
    \item the class list and integer-to-label mapping.
\end{itemize}
This ensures that deployment uses identical preprocessing and label semantics,
avoiding training--serving skew.

% ---------------------------------------------------------------------------
\section{Deployment: Backend Service and Web Frontend}
\label{sec:deployment}

The final system integrates the analytical selector and trained classifier into
an interactive decision-support tool.

\subsection{Backend Architecture}
\label{subsec:backend_architecture}

The backend exposes a REST API built with FastAPI. On each request, the service:
\begin{enumerate}
    \item validates input fields against the same constraints used during
          dataset generation,
    \item invokes the analytical selector to compute capacities, feasibility
          flags, design torque, and scores,
    \item assembles features and runs the persisted ML pipeline to obtain a
          predicted label and class probabilities,
    \item returns a unified response containing both analytical and ML outputs
          plus diagnostic fields (friction used, hub stiffness factor,
          interference results).
\end{enumerate}
The response schema mirrors the dataset columns, reducing training--serving skew.

\subsection{Frontend Interface}
\label{subsec:frontend_interface}

The React-based frontend provides input controls for geometry, materials,
operating conditions, and the eight preference sliders. Client-side validation
mirrors backend constraints to prevent invalid submissions. After submission,
the UI displays:
\begin{itemize}
    \item the analytical recommendation with torque capacities and feasibility
          status,
    \item scores across all feasible candidates for interpretability,
    \item the ML prediction label and probability distribution for transparency.
\end{itemize}

Presenting both outputs side-by-side fulfills the thesis objective: mechanical
consistency is preserved through explicit capacity computations, while rapid
probabilistic recommendations are available through the trained classifier.
Disagreements between the two sources can prompt users to examine edge cases
more carefully.

% ---------------------------------------------------------------------------
\section{Summary}
\label{sec:method_summary}

This chapter presented the hybrid methodology underlying the shaft--hub
connection selector. A material database and DIN-based geometry tables provide
the foundation for capacity calculations covering press fits, keys, and splines.
The analytical selector applies feasibility filters---including a
manufacturability check for press-fit interference---and ranks feasible
candidates using preference-weighted scoring. This deterministic engine then
labels a synthetic dataset generated by sampling realistic input distributions.
Supervised classifiers are trained on the synthetic data, with macro F1-score
guiding model selection. The selected model is persisted with full metadata and
integrated into a FastAPI backend alongside the analytical selector. A React
frontend presents both analytical and ML recommendations, delivering transparent
and interpretable decision support for shaft--hub connection selection.
