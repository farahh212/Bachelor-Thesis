\chapter{Conclusion}
\label{ch:conclusion}

This chapter summarizes the work presented in this thesis, discusses its relation to existing work, addresses limitations, outlines directions for future research, and provides concluding remarks.

% ---------------------------------------------------------------------------
\section{Research Objectives Revisited}
\label{sec:objectives_revisited}

This thesis successfully addressed all five research objectives defined in Chapter~1:

\begin{enumerate}
    \item \textbf{Develop a scoring system pipeline:} A comprehensive analytical scoring system was developed that assigns shaft--hub connection labels based on input parameters and preference-driven criteria. The system incorporates torque-based feasibility rules, manufacturability checks, and preference-weighted evaluation across eight application dimensions.

    \item \textbf{Scale pipeline to generate synthetic dataset:} The scoring pipeline was successfully scaled to generate a large and diverse synthetic dataset containing 4{,}993 samples. Input parameters were randomized in accordance with DIN standards, covering diameters from 6~mm to 230~mm, torque requirements from 103~N$\cdot$m to 13.6~MN$\cdot$m, and diverse material combinations.

    \item \textbf{Train and evaluate ML classification model:} Multiple machine-learning classifiers were trained and evaluated on the synthetic dataset. CatBoost achieved the highest macro F1-score of 0.7986 and was selected as the production model, demonstrating balanced performance across all three connection classes (press fit: F1=0.6774, key: F1=0.8057, spline: F1=0.9127).

    \item \textbf{Integrate model into web application:} The trained model was successfully integrated into a web-based application using FastAPI backend and React frontend. The application provides real-time connection recommendations with both analytical and ML outputs displayed side-by-side.

    \item \textbf{Present analytical capacities and ML confidence scores:} The system presents analytical torque capacities for all connection types alongside ML softmax-based confidence scores, improving transparency and interpretability. Users can compare mechanical feasibility with probabilistic predictions, enabling informed decision-making.
\end{enumerate}

All objectives were met, resulting in a fully functional decision-support tool that combines analytical rigor with machine-learning efficiency.

% ---------------------------------------------------------------------------
\section{Summary}
\label{sec:summary}

This thesis developed a Hybrid Analytical–Machine Learning framework for intelligent selection of shaft–hub connections, combining analytical mechanical models with machine learning. Analytical torque capacity models for three connection types (press fits, keys, splines) were implemented based on DIN standards~\cite{DIN7190_1_2017,DIN6885_1_2021,DIN5480_1_2006}, including feasibility checks and preference-weighted scoring. A synthetic dataset was generated, enabling ML training in the absence of empirical data~\cite{Picard_2023}. The ML model was integrated with analytical calculations in a web application, creating an explainable AI tool that accelerates decision-making while providing insight into underlying mechanics.

% ---------------------------------------------------------------------------
\section{Relation to Existing Work}
\label{sec:related_work}

Traditionally, shaft–hub connection selection has been a manual process relying on charts and engineer judgment~\cite{Haggenmueller_2025}. This thesis contributes a holistic approach, showing how to build training data in silico using engineering knowledge when public datasets don't exist~\cite{Picard_2023}. The work underscores the importance of maintaining mechanical consistency in AI applications for engineering~\cite{Raissi_Perdikaris_Karniadakis_2019,Karniadakis_Kevrekidis_Lu_Perdikaris_Wang_Yang_2021}, ensuring AI predictions never violate fundamental engineering constraints while maintaining interpretability and trust.

% ---------------------------------------------------------------------------
\section{Limitations and Future Work}
\label{sec:future_work}

Several limitations must be acknowledged. First, the analytical models are simplified and mostly static; dynamic factors such as fatigue life, impact loads, and long-term wear are not included. Recommendations are suitable for initial design decisions but should be further validated for durability. Second, the system considers only three connection types; extending to additional types (tapered shrink disks, polygonal shafts, etc.) would require implementing their analytical models and retraining the ML model. Third, the user preference interface uses linear weighting that does not capture interactions; future work could explore more sophisticated multi-criteria decision-making techniques such as AHP (e.g., asking users to compare criteria pairwise, such as whether durability is more important than cost, and converting these judgments into consistent weights to rank the connection alternatives). Fourth, implementing uncertainty quantification would help identify when the model extrapolates beyond its training data, for example when input designs fall outside the range of materials, loads, or geometries used during training. In such cases, high uncertainty could be used to flag recommendations as low-confidence and prompt fallback to analytical checks or user review. Finally, real-world validation through industry case studies or classroom testing would provide valuable feedback, potentially extending the system to suggest specific dimensions and tolerances, not just connection types.

% ---------------------------------------------------------------------------
\section{Outlook}
\label{sec:outlook}

Future modifications to the ML program could significantly enhance its precision and applicability. Several promising directions are outlined below.

\subsection{Integration of Literature and Research Data}

A key enhancement would be the integration of automated literature scanning and research article analysis capabilities. Natural language processing (NLP) techniques, such as named entity recognition and information extraction, could be employed to automatically extract relevant design parameters, material properties, and empirical performance data from engineering textbooks, research papers, and technical documentation. This would enable the system to continuously learn from published knowledge, incorporating real-world case studies and experimental results that may not be fully captured by analytical models alone.

The extracted information could be used in several ways: (1) as additional training data to augment the synthetic dataset, particularly for boundary cases and edge conditions; (2) as validation benchmarks to verify analytical model predictions against experimental evidence; and (3) as correction factors to refine torque capacity equations based on empirical findings. For instance, if research articles consistently report that certain material combinations exhibit different friction coefficients than those assumed in DIN standards, the system could incorporate these corrections automatically.

\subsection{Enhanced Precision Through Multi-Source Learning}

The integration of multiple knowledge sources, analytical models, literature data, experimental databases, and historical design records, could create a more robust and precise prediction system. A multi-source learning framework could weight different information sources based on their reliability and relevance, with analytical models providing the foundation, literature providing empirical corrections, and historical data providing context-specific insights. This approach would address the current limitation of relying solely on synthetic data derived from standards, while maintaining the interpretability and physical consistency of the analytical framework.

\subsection{Continuous Learning and Model Updates}

The ML program could be designed to support continuous learning, where new data from literature scans, user feedback, and real-world applications are periodically incorporated into the model. This would require careful versioning and validation procedures to ensure that updates improve rather than degrade performance. A feedback loop mechanism could allow engineers to report actual performance outcomes, which could then be used to refine both the analytical models and the ML classifier, creating a self-improving system that becomes more accurate over time.

\subsection{Technical Implementation Considerations}

Implementing literature scanning capabilities would require: (1) access to digital libraries and research databases (e.g., IEEE Xplore, ASME Digital Collection, SpringerLink); (2) text mining pipelines capable of extracting structured engineering data from unstructured text; (3) quality control mechanisms to filter out erroneous or outdated information; and (4) integration frameworks to merge extracted data with existing analytical models. While technically feasible, this would represent a significant extension of the current system, requiring expertise in NLP, knowledge engineering, and database management. The potential benefits, however, could substantially improve the system's precision and make it a more comprehensive design support tool.

% ---------------------------------------------------------------------------
\section{Concluding Remarks}
\label{sec:concluding_remarks}

In conclusion, this thesis has demonstrated that a hybrid approach combining analytical engineering models with machine learning can effectively automate design decisions in a manner that is both efficient and trustworthy. By encoding domain knowledge from standards into a form that AI can learn, the gap between explicit knowledge and data-driven inference was bridged. The developed system provides engineers with a powerful assistant that recommends solutions while explaining them, preserving interpretability and confidence essential in engineering applications.

This work contributes to the broader vision of intelligent CAD/CAE tools where routine decisions are augmented by AI. The methodology of generating synthetic data from analytical models and using it to train ML classifiers is broadly applicable and could be used to create similar decision-support tools in other design domains. This thesis illustrates the potential for improved design workflows that are faster yet remain mechanically sound and explainable.
