\chapter{Conclusion}
\label{ch:conclusion}

This chapter summarizes the work presented in this thesis, discusses its relation to existing work, addresses limitations, outlines directions for future research, and provides concluding remarks.

% ---------------------------------------------------------------------------
\section{Summary}
\label{sec:summary}

This thesis developed a Hybrid Analytical–Machine Learning framework for intelligent selection of shaft–hub connections. The core idea was to combine the reliability of analytical mechanical models with the adaptability of machine learning to handle multi-criteria decision-making. Analytical torque capacity models for three connection types---press fits, keys, and splines---were implemented based on authoritative standards (DIN~7190, DIN~6885, DIN~5480) and engineering best practices~\cite{DIN7190_2017,DIN6885_2021,DIN5480_2006}. These models included detailed checks for feasibility (e.g., ensuring press fit pressures and interferences remain within allowable limits and that key stresses do not exceed material strengths). On top of this, a novel preference-weighted scoring system was built that quantifies subjective design considerations, allowing the framework to tailor its recommendation to specific project priorities.

Using this combined analytical logic, a comprehensive synthetic dataset of design cases was generated, which enabled training a classification ML model in the absence of any prior empirical data~\cite{Picard_2023}. The machine learning model---an ensemble of tree-based classifiers~\cite{Breiman_2001}---learned to predict the optimal connection type with high accuracy, essentially generalizing the rules embedded in the analytical scorer. The ML model was integrated with the analytical calculations in a web application, creating a user-friendly interface where a designer can input parameters and immediately receive a recommended connection along with explanatory data (capacities, safety margins, confidence levels). The final system successfully demonstrates an explainable AI tool for engineering design: one that accelerates decision-making while still providing insight into the underlying mechanics.

Key outcomes include: (1) a working decision-support application for shaft–hub connections that covers a wide range of sizes and use cases; (2) a validated approach to generating and using synthetic engineering datasets for ML, which can serve as a template for other domains lacking large datasets; and (3) evidence that hybridizing analytical models with ML can yield robust and user-accepted solutions, combining domain knowledge with data-driven prediction. The approach effectively bridges deterministic and data-driven methods, showing they are complementary rather than conflicting in the context of engineering design problems.

% ---------------------------------------------------------------------------
\section{Relation to Existing Work}
\label{sec:related_work}

Traditionally, shaft–hub connection selection has been a manual process relying on charts (such as those found in machine design textbooks) and the engineer's judgment~\cite{Haggenmueller_2025}. Prior research in mechanical design automation has often focused on specific aspects---for example, optimizing interference fit parameters or analyzing stress in keyed joints---but not on an end-to-end selection among different connection types. This thesis contributes a unique holistic approach. It draws inspiration from previous studies that employed machine learning in engineering, which typically require datasets gleaned from either experimental data or large databases of past designs. In the shaft–hub context, such data did not exist publicly. This work has shown how to build that data in silico using engineering knowledge. This represents an advancement in the methodology of engineering design: rather than seeing the lack of data as a barrier, standards and simulation were leveraged to create the necessary dataset. Recent works in adjacent fields (e.g., ML for material selection or gear design) have also begun to use synthetic data~\cite{Picard_2023}, and this work reinforces the validity of that approach for intelligent design systems.

In terms of practical design tools, this system can be seen as an evolution of calculation software that incorporates selection logic. For instance, some commercial software can calculate press fit safety or suggest a fit based on DIN standards, but they do not integrate user preferences or learning. This framework could inform future CAD/CAE software features where an AI assistant actively helps in component selection by learning from built-in engineering libraries.

This work also underscores the importance of maintaining mechanical consistency in AI applications for engineering. Unlike end-to-end black-box models sometimes seen in research, this thesis ensured that the AI's predictions never violate fundamental engineering constraints---a key requirement for acceptance in safety-critical fields. This aligns with the emerging perspective in the literature that combining physics-based models with machine learning (sometimes termed physics-informed ML) yields better trust and often better accuracy. The ML model's structure and training were explicitly rooted in known engineering relations, thus positioning this system as a step towards ``expert systems 2.0''---where expert knowledge guides an AI, rather than the AI operating independently.

% ---------------------------------------------------------------------------
\section{Limitations and Future Work}
\label{sec:future_work}

While the developed system performs well within its defined scope, several limitations must be acknowledged. First, the analytical models used are simplified and mostly static. Dynamic factors such as fatigue life, impact loads, and long-term wear are not included. This means the recommendations are suitable for initial design decisions but should be further validated for durability. Including a fatigue-life estimation (perhaps via an additional analysis for each connection type) could be a valuable extension. Second, the system currently considers only three connection types. Other shaft–hub connections exist (tapered shrink disks, polygonal shafts, cross-pin connections, etc.) that might be relevant in certain industries. Extending the framework to additional connection types is straightforward---it would require implementing their analytical models and adding those to the candidate set. The ML model could then be retrained on an expanded dataset. For example, including a clamping element (shrink disk) as a frictional connection option could cover cases where a truly removable high-torque connection is needed; the current system might suggest a spline for that, but a shrink disk might actually be preferable.

Another area for improvement is the user preference interface. The current linear weighting is straightforward but does not capture interactions (e.g., a user might care about cost only if maintenance frequency is low). Future work could explore more sophisticated multi-criteria decision-making techniques, such as the Analytical Hierarchy Process (AHP) or trade-off diagrams, to let users express preferences in a more nuanced way. Additionally, one could attempt to learn preference weightings by observing decisions made by experts, thereby tuning the scoring system to mimic expert choices in historical cases.

On the machine learning side, an interesting future direction is implementing uncertainty quantification. Even though the model outputs probabilities, these are more akin to confidence and not true epistemic uncertainty. Techniques such as Bayesian neural networks or ensemble variance could give a sense of when the model is extrapolating beyond its training (and thus warn the user or fall back on pure analytical computation). This could be important if, for instance, a user inputs a combination far outside normal design ranges; the system should ideally recognize ``I have not seen this before'' and be cautious.

Finally, it would be beneficial to conduct real-world validation of the system. This could involve testing the tool on case studies from industry or in a classroom setting. Feedback from practicing engineers would likely provide insights on additional factors to include. For example, they might want the system to also suggest dimensions (not just type)---e.g., recommend a certain interference value or a specific DIN~5480 spline size. That would shift it from just selecting type to sizing the connection, which is a natural next step. The current framework lays the groundwork for that: the analytical model already computes what interference is needed, etc., so extending the output to recommend specific tolerances or fits (such as ``Use an H7/p6 interference fit with $25~\mu\text{m}$ interference'') could be achieved.

% ---------------------------------------------------------------------------
\section{Concluding Remarks}
\label{sec:concluding_remarks}

In conclusion, this thesis has demonstrated that a hybrid approach combining analytical engineering models with machine learning can effectively automate a design decision process---in this case, selecting shaft–hub connections---in a manner that is both efficient and trustworthy. By encoding the domain knowledge from standards into a form that an AI can learn, the gap between explicit knowledge and data-driven inference was bridged. The developed system provides engineers with a powerful assistant that not only recommends a solution but also explains it, thus preserving the interpretability and confidence that are essential in engineering applications.

This work contributes to the broader vision of intelligent CAD/CAE tools where routine decisions are augmented by AI, allowing engineers to focus creativity and expertise on the more complex aspects of design. The methodology of generating synthetic data from analytical models and using it to train ML classifiers is broadly applicable and could be used to create similar decision-support tools in other design domains (for example, material selection, bearing selection, etc.). As engineering practice increasingly incorporates AI-based tools, data-driven design is expected to become more prevalent, with machine learning algorithms embedded in engineering software to provide instant recommendations that adhere to proven rules. This thesis is a step in that direction, illustrating the potential for improved design workflows that are faster yet remain mechanically sound and explainable. Ultimately, the synergy of human expertise, rigorous standards, and artificial intelligence can lead to better designs achieved in less time---a significant advantage in the competitive field of mechanical product development.
