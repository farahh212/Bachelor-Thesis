% ============================================================================
% TAS SCHAFER INTEGRATION SECTIONS
% Add these sections to your thesis as indicated below
% ============================================================================

% ----------------------------------------------------------------------------
% SECTION 1: Add to Background (Chapter 2)
% Location: After section on press fits, keys, and splines
% ----------------------------------------------------------------------------

\subsection{Commercial Shaft--Hub Connection Solutions}
\label{subsec:commercial_solutions}

While this thesis focuses on three fundamental connection types (press fits, 
keys, and splines) based on DIN standards, commercial solutions exist that 
extend these concepts to achieve higher torque capacities or additional 
functionality. TAS Schafer, a leading manufacturer of shaft--hub connections, 
offers two product categories that combine interference fits with clamping 
elements: \emph{shrink discs} and \emph{locking assemblies}.

\textbf{Shrink Discs} utilize an interference fit base enhanced by radial 
clamping elements (typically screws) that apply additional pressure at a 
working diameter $d_w$ smaller than the shaft diameter $d$. This hybrid 
approach combines the benefits of interference fits (friction-based torque 
transmission) with the adjustability and higher capacity of clamping 
mechanisms. The torque capacity $M_t$ of a shrink disc is the sum of two 
contributions:
\begin{equation}
M_{t,\text{shrink disc}} = M_{t,\text{interference}} + M_{t,\text{clamping}},
\end{equation}
where $M_{t,\text{interference}}$ follows the standard press-fit relationship 
and $M_{t,\text{clamping}}$ depends on the number of clamping elements $Z$, 
assembly torque $M_A$, and clamping geometry parameters.

\textbf{Locking Assemblies} employ a pure clamping mechanism without 
interference fits, providing both torque transmission and axial force capacity 
$F_{ax}$ through controlled clamping element pressure. Unlike shrink discs, 
locking assemblies do not rely on interference fits, making them suitable for 
applications requiring disassembly or adjustment.

These commercial solutions represent an extension of the fundamental 
connection types considered in this thesis. While they offer advantages 
in specific applications (higher capacity, adjustability, axial force 
transmission), their integration into the current framework would require:
\begin{itemize}
    \item Proprietary analytical models (not standardized in DIN standards)
    \item Additional geometric parameters (clamping elements $Z$, working 
          diameter $d_w$, assembly torque $M_A$)
    \item Extended feature engineering (handling missing catalog data such 
          as user preferences and explicit material specifications)
    \item Significant pipeline modifications (analytical models, dataset 
          generation, ML model retraining)
\end{itemize}

For the scope of this thesis, the focus remains on the three fundamental 
types (press fits, keys, splines) that form the basis of most shaft--hub 
connections and are fully standardized in DIN~7190, DIN~6885, and DIN~5480. 
The framework's architecture, however, supports extension to additional 
connection types, as discussed in Section~\ref{sec:future_work}.

% ----------------------------------------------------------------------------
% SECTION 2: Add to Discussion (Chapter 5)
% Location: In limitations/extensibility discussion
% ----------------------------------------------------------------------------

\paragraph{Extensibility to Additional Connection Types.}
The developed framework demonstrates a clear pathway for extending to 
additional connection types beyond the three fundamental types considered. 
The methodology of encoding analytical models into synthetic data generation 
and training ML classifiers is directly applicable to other connection types, 
such as commercial shrink discs or locking assemblies.

The key requirements for such extensions are:
\begin{enumerate}
    \item \textbf{Analytical Model Development:} Implement analytical models 
          for the new connection type(s), accounting for their specific 
          physics (e.g., clamping element contributions, working diameter 
          effects, axial force capacity).
    
    \item \textbf{Parameter Sampling Strategies:} Develop sampling strategies 
          aligned with the new type's design space, maintaining DIN-compliant 
          or manufacturer-specified parameter ranges.
    
    \item \textbf{Feature Engineering:} Extend the feature set to include 
          new parameters (e.g., clamping elements, assembly torque) and handle 
          missing features in catalog data through defaults or estimates.
    
    \item \textbf{Synthetic Data Generation:} Generate synthetic training 
          data using the new analytical models, following the same automated 
          pipeline approach used for the three fundamental types.
    
    \item \textbf{Model Retraining:} Retrain the ML classifier with extended 
          classes and features, ensuring no degradation in performance on 
          the original connection types.
\end{enumerate}

While this thesis focuses on three fundamental types, the framework's 
modular architecture supports such extensions, as demonstrated by the 
separation of analytical models, dataset generation, and ML training 
components. The hybrid analytical--ML approach ensures that new connection 
types can be integrated while maintaining mechanical consistency and 
interpretability.

% ----------------------------------------------------------------------------
% SECTION 3: Add to Future Work (Chapter 6)
% Location: Extend existing future work section
% ----------------------------------------------------------------------------

\paragraph{Extension to Commercial Connection Types.}
The current framework considers three fundamental connection types based 
on DIN standards. A natural extension would integrate commercial solutions 
such as TAS Schafer shrink discs and locking assemblies, which combine 
interference fits with clamping elements to achieve higher torque capacities.

\textbf{Integration Requirements:}

\begin{enumerate}
    \item \textbf{Analytical Model Development:} Implement analytical models 
          for shrink disc and locking assembly torque capacity, accounting 
          for clamping element contributions and working diameter effects. 
          This requires studying proprietary equations from manufacturer 
          documentation and validating against catalog data. The shrink disc 
          model must combine interference fit and clamping contributions, 
          while the locking assembly model must handle pure clamping 
          mechanisms and axial force capacity.
    
    \item \textbf{Feature Engineering:} Extend the feature set to include 
          clamping-specific parameters (number of clamping elements $Z$, 
          assembly torque $M_A$, working diameter $d_w$) and handle missing 
          features in catalog data (user preferences, explicit materials) 
          through defaults or estimates. The feature set would expand from 
          18 to approximately 28--30 features.
    
    \item \textbf{Synthetic Data Generation:} Generate synthetic training 
          data for TAS types using the new analytical models, maintaining 
          the same DIN-compliant sampling approach used for the three 
          fundamental types. This would generate approximately 2,000--3,000 
          synthetic shrink disc samples and 1,000--1,500 synthetic locking 
          assembly samples to balance the dataset.
    
    \item \textbf{Model Extension:} Retrain the ML classifier to handle 
          five classes (press, key, spline, shrink disc, locking assembly) 
          with extended features, ensuring no degradation in performance on 
          the original three classes. This requires careful handling of 
          class imbalance, as TAS types would represent approximately 50\% 
          of the combined dataset.
    
    \item \textbf{System Integration:} Extend the web application to support 
          TAS-specific input fields (clamping elements, assembly torque, 
          working diameter) and display recommendations for all five connection 
          types, with appropriate UI toggles for enabling TAS types.
\end{enumerate}

\textbf{Estimated Effort:} 8--11 weeks of development time, including 
analytical model implementation (2--3 weeks), dataset generation and feature 
engineering (2--3 weeks), model retraining (1 week), system integration 
(2 weeks), and testing/validation (1 week).

\textbf{Challenges:}
\begin{itemize}
    \item Proprietary equations require reverse-engineering from catalog 
          data and manufacturer documentation, as they are not standardized 
          in DIN standards.
    
    \item Different physics for each type: shrink discs combine interference 
          fits with clamping, while locking assemblies use pure clamping 
          mechanisms, requiring separate analytical models.
    
    \item Class imbalance: TAS types would represent approximately 50\% of 
          the combined dataset, potentially affecting model performance on 
          the original three classes.
    
    \item Feature incompatibility: Catalog data lacks user preferences and 
          explicit material specifications, requiring feature engineering with 
          defaults or estimates.
    
    \item Validation complexity: Proprietary models must be validated against 
          manufacturer catalog data, which may have simplified or proprietary 
          assumptions not fully documented.
\end{itemize}

This extension would demonstrate the framework's extensibility and provide 
a more comprehensive solution covering both standardized (DIN-based) and 
commercial connection types, while maintaining the hybrid analytical--ML 
approach that ensures mechanical consistency and interpretability.

% ============================================================================
% END OF TAS INTEGRATION SECTIONS
% ============================================================================


