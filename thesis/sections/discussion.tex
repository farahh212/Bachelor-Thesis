\chapter{Discussion}
\label{ch:discussion}

This chapter discusses the findings of the developed hybrid analytical–machine learning framework, examining the behavior of the analytical model, the effect of preference-weighted scoring, the integration of synthetic data with machine learning, and overall system-level considerations and limitations.

% ---------------------------------------------------------------------------
\section{Analytical Model Behavior and Validity}
\label{sec:analytical_behavior}

The hybrid approach ensures that no mechanically unfit solution is ever recommended, because the analytical feasibility check acts as a gatekeeper~\cite{Raissi_Perdikaris_Karniadakis_2019,Karniadakis_Kevrekidis_Lu_Perdikaris_Wang_Yang_2021}. Analysis demonstrates consistency with standard engineering knowledge~\cite{DIN7190_1_2017,DIN6885_1_2021,DIN5480_1_2006,Budynas_Nisbett_2020}. The interference plausibility filter for press fits proved important: it prevented the model from favoring press fits in scenarios where required interference was unrealistically high, correctly defaulting to splines or keys, which aligns with practical design rules where press fits are limited by assembly constraints~\cite{FVA_InterferenceFits_2017}.

For keyed joints, the bearing pressure limit often governs rather than shear strength, especially for larger shafts or softer hub materials, matching standard design practice~\cite{DIN6885_1_2021}. Keyed connections can fail through two distinct mechanisms: (a) \textbf{shear failure}, where the key is cut across by torque, controlled by the key's cross-sectional area and shear strength; and (b) \textbf{bearing (crushing) failure}, where the key presses against the side of the hub keyway, causing plastic deformation of the hub or key, keyway wall damage, and potential loosening and backlash. Bearing pressure typically governs for two primary reasons. First, torque scales approximately with the cube of shaft diameter ($T \propto d^3$), while key dimensions increase only linearly with diameter according to DIN~6885 standards. This means that as shaft size increases, the contact area between the key and keyway does not increase fast enough to accommodate the higher torque, causing bearing pressure to rise faster than shear stress. Second, many hub materials (cast iron, aluminum alloys, mild steel) have lower compressive yield strength and allowable bearing stress compared to the key material, so the hub deforms first rather than the key shearing. The analytical model calculates both shear and bearing capacities and uses the minimum (governing failure mode) to determine torque capacity, ensuring conservative design according to DIN~6885 principles.

Splined connections showed very high torque capacity across a wide range of sizes due to load sharing across multiple teeth~\cite{DIN5480_1_2006}. The system did not always select the spline even if it had far more capacity, by design, to avoid unnecessary overdesign. The overdesign penalty ensures that simpler solutions are preferred when adequate, reflecting engineering economy principles.

A limitation of the analytical model is its use of simplified assumptions (e.g., fixed friction coefficients). Several factors that affect real-world performance are not fully captured. \textbf{Stress concentrations in keyed joints:} Keyways introduce geometric discontinuities (sharp corners, transitions between shaft and keyway), which cause local stress amplification at keyway fillets and reduced fatigue strength. These stress concentrations can lead to early crack initiation under cyclic loading, even when nominal stresses appear acceptable. The analytical model uses nominal stresses based on DIN~6885 and does not explicitly account for these local peak stresses, which is acceptable for static design but may underestimate fatigue risk. \textbf{Fretting in press fits under cyclic loads:} Under fluctuating torque or bending, microscopic slip can occur at the press-fit interface, leading to fretting wear and fatigue cracks that reduce effective friction and joint life over time. Analytical models typically assume perfect sticking (no micro-slip) and no surface degradation, which is reasonable for static design but may overestimate long-term capacity under dynamic loading. The tool is conservative within its scope but cannot replace detailed analysis for final design. More detailed validation using finite element analysis and experimental measurements~\cite{Cajuhi_Pepe_Moreno_2009} could provide additional confidence for critical applications, particularly those involving fatigue or dynamic loading.

% ---------------------------------------------------------------------------
\section{Effect of Preference-Weighted Scoring}
\label{sec:preference_scoring}

Preference weighting allows the system to differentiate between multiple feasible solutions in a rational and traceable manner. When both a press fit and a spline are feasible, the one aligning better with user priorities is recommended. This provides a quantitative voice to non-strength criteria, making the tool context-aware. If contradictory preferences are set (e.g., maximum on both low cost and high durability), the system weights them equally, which may lead to balanced decisions. The linear weighted sum approach worked well in tests but could be refined using more advanced multi-criteria decision-making methods. The scoring mechanism includes built-in penalties for scenarios such as press fits in thin hubs, which would be difficult to capture in a pure ML model.

% ---------------------------------------------------------------------------
\section{Synthetic Data and ML Model Integration}
\label{sec:synthetic_data_integration}

Using synthetic data generated from analytical rules proves successful for training the ML model~\cite{Picard_2023}. The ML model serves as a fast surrogate and generalizes decisions in a smoother manner. The boundaries learned can effectively interpolate between cases, providing probabilistic transitions (e.g., 0.4 probability to key and 0.6 to spline) rather than abrupt switches, naturally expressing confidence levels. From a performance standpoint, the ML model enables scalability for optimization loops, providing near-instant decisions without repeated heavy calculations.

The ML model's validity is tied to synthetic data quality and inherits the analytical generator's biases and limitations. If a query falls outside the original parameter range, the model might be less accurate. The integration with the analytical backend means the analytical check always serves as a safety net, ensuring recommendations remain feasible. Testing reveals no contradictions; ML outputs usually correspond to feasible options, adding trustworthiness.

% ---------------------------------------------------------------------------
\section{System-Level Considerations and Limitations}
\label{sec:limitations}

The combined system demonstrates a pathway for AI-assisted engineering design, showing that machine learning can be harnessed without experimental data by using established knowledge to generate synthetic data. A key benefit is interpretability: users see not only what the ML predicts, but why, because underlying physics are exposed (torque margins, etc.). This addresses reluctance to trust AI in critical engineering decisions.

Limitations include: the model assumes static loads and single operating conditions; fatigue, shock loads, misalignment, and environmental influences are not explicitly modeled (though some are partially accounted for via preferences); the material library is limited; and friction coefficient distributions are simplified. Future work could integrate dynamic factors and expand material coverage.

The FastAPI/React setup illustrates that engineering tools can be made accessible via web technology. The tool is well-suited for educational use or preliminary design, but final decisions should be reviewed by experts or verified with detailed analysis, especially for critical applications~\cite{Haggenmueller_2025}.

A broader implication is the demonstrated feasibility of encoding engineering standards into a format that AI can learn. Parts of DIN~7190, 6885, and 5480 were effectively translated into a dataset and model~\cite{DIN7190_1_2017,DIN6885_1_2021,DIN5480_1_2006}. This approach could be extended to other design standards, suggesting a future where engineers have AI assistants for different design decisions, all grounded by domain knowledge of standards.
