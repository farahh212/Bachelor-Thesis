\chapter{Results}
\label{ch:results}

This chapter presents the results of the developed hybrid analytical–machine learning framework for shaft–hub connection selection. The results are organized into six sections: requirements validation mapping test results to methodology requirements, analytical model verification, synthetic dataset characteristics, machine learning model performance including error analysis and statistical significance testing, and a demonstration of the integrated web application.

% ---------------------------------------------------------------------------
\section{Analytical Model Verification}
\label{sec:analytical_verification}

To validate the analytical model's accuracy, three standardized test cases were evaluated against ground truth values computed using DIN standards (DIN~7190 for press fits, DIN~6885 for keys, and DIN~5480 for splines). The test configuration used a $45~\text{mm}$ solid shaft with $50~\text{mm}$ hub length, Steel E360 material for thr shaft and Steel 16MnCr5 for the hub, required torque of $870~\text{N}\cdot\text{m}$ ($870{,}000~\text{N}\cdot\text{mm}$), and a safety factor of 2.0. Table~\ref{tab:test_case_inputs} summarizes the common input parameters, while Tables~\ref{tab:test_press_fit},~\ref{tab:test_key}, and~\ref{tab:test_spline} present detailed comparisons for each connection type, including geometrical and material data, analytical results, and predicted torque capacities from the program.

\begin{table}[ht]
\centering
\caption{Common input parameters for standardized test cases}
\label{tab:test_case_inputs}
\begin{tabular}{lc}
\toprule
Parameter & Value \\
\midrule
Shaft diameter $d$ & $45~\text{mm}$ \\
Hub length $L$ & $50~\text{mm}$ \\
Shaft material & Steel E360 \\
Hub material & Steel 16MnCr5 \\
Required torque $M_{\text{req}}$ & $870~\text{N}\cdot\text{m}$ ($870{,}000~\text{N}\cdot\text{mm}$) \\
Safety factor $S_R$ & 2.0 \\
Shaft type & Solid \\
\bottomrule
\end{tabular}
\end{table}

\subsubsection{Press Fit Test Case}
\label{subsubsec:test_press_fit}

The press fit test case used a hub outer diameter of $70~\text{mm}$ and a friction coefficient $\mu = 0.2$. Table~\ref{tab:test_press_fit} presents the geometrical and material parameters, analytical results, and predicted torque capacity from the program. The model correctly computed the required interface pressure ($p_{\text{erf}} = 54.7~\text{MPa}$) and allowable pressure ($p_{\text{zul}} = 124.2~\text{MPa}$), both matching DIN~7190 ground truth values within $0.1\%$. The torque capacity from allowable pressure ($M_{t,\text{zul}} = 3{,}951~\text{N}\cdot\text{m}$) exceeds the required torque, indicating mechanical feasibility. However, the interference check correctly identified that the working interference $U_w = 0.0316~\text{mm}$ exceeds the practical limit of $0.020~\text{mm}$ for this diameter range, rendering the press fit infeasible from a manufacturability perspective. This demonstrates the model's ability to enforce practical design constraints beyond pure mechanical capacity.

\begin{table}[ht]
\centering
\caption{Press fit test case: geometrical/material data, analytical results, and predicted torque. Ground truth values from~\cite{Result_SHC2_2024}.}
\label{tab:test_press_fit}
\begin{tabular}{lc}
\toprule
Parameter & Value \\
\midrule
\multicolumn{2}{l}{\textbf{Geometrical and Material Data}} \\
\midrule
Shaft diameter $d$ & $45~\text{mm}$ \\
Hub outer diameter $D$ & $70~\text{mm}$ \\
Hub length $L$ & $50~\text{mm}$ \\
Shaft material & Steel E360 \\
Hub material & Steel 16MnCr5 \\
Friction coefficient $\mu$ & $0.2$ \\
Required torque $M_{\text{req}}$ & $870~\text{N}\cdot\text{m}$ \\
Safety factor $S_R$ & $2.0$ \\
\midrule
\multicolumn{2}{l}{\textbf{Analytical Results}} \\
\midrule
Diameter ratio $Q_A$ & $0.643$ \\
Required pressure $p_{\text{erf}}$ & $54.7~\text{MPa}$ \\
Allowable pressure $p_{\text{zul}}$ & $124.2~\text{MPa}$ \\
Elastic interference $U_e$ & $0.0412~\text{mm}$ \\
Working interference $U_w$ & $0.0316~\text{mm}$ \\
Interference limit & $0.020~\text{mm}$ \\
\midrule
\multicolumn{2}{l}{\textbf{Predicted Torque from Program}} \\
\midrule
Torque capacity $M_{t,\text{zul}}$ & $3{,}951~\text{N}\cdot\text{m}$ \\
Feasibility (mechanical) & Yes \\
Feasibility (practical) & No ($U_w > 0.020~\text{mm}$) \\
\bottomrule
\end{tabular}
\end{table}

\subsubsection{Key Test Case}
\label{subsubsec:test_key}

The key test case used standard DIN~6885 key dimensions: width $b = 14~\text{mm}$ and height $h = 9~\text{mm}$ for a $45~\text{mm}$ shaft diameter. Table~\ref{tab:test_key} presents the geometrical and material parameters, analytical results, and predicted torque capacity from the program. The model correctly identified the key dimensions and computed a torque capacity of $945~\text{N}\cdot\text{m}$ ($945{,}000~\text{N}\cdot\text{mm}$), which matches the DIN~6885 ground truth value of $944~\text{N}\cdot\text{m}$ within $0.1\%$. However, when the safety factor $S_R = 2.0$ is applied, the corresponding design torque is $M_{\text{design}} = 1{,}740~\text{N}\cdot\text{m}$, which exceeds the key capacity. In the analytical selector, this key configuration is therefore marked as \emph{mechanically infeasible} for the standardized test case, even though the underlying torque formula itself agrees closely with the DIN reference.

\begin{table}[ht]
\centering
\caption{Key test case: geometrical/material data, analytical results, and predicted torque. Ground truth values from~\cite{Result_SHC2_2024}.}
\label{tab:test_key}
\begin{tabular}{lc}
\toprule
Parameter & Value \\
\midrule
\multicolumn{2}{l}{\textbf{Geometrical and Material Data}} \\
\midrule
Shaft diameter $d$ & $45~\text{mm}$ \\
Key width $b$ & $14~\text{mm}$ \\
Key height $h$ & $9~\text{mm}$ \\
Key length $L$ & $50~\text{mm}$ \\
Shaft material & Steel E360 \\
Hub material & Steel 16MnCr5 \\
Required torque $M_{\text{req}}$ & $870~\text{N}\cdot\text{m}$ \\
Safety factor $S_R$ & $2.0$ \\
\midrule
\multicolumn{2}{l}{\textbf{Analytical Results}} \\
\midrule
Allowable shear stress $\tau_{\text{zul}}$ & $60~\text{MPa}$ \\
Allowable bearing pressure $p_{\text{zul}}$ & $333~\text{MPa}$ \\
\midrule
\multicolumn{2}{l}{\textbf{Predicted Torque from Program}} \\
\midrule
Torque capacity $M_{t,\text{zul}}$ & $945~\text{N}\cdot\text{m}$ \\
Ground truth (DIN~6885) & $944~\text{N}\cdot\text{m}$ \\
Deviation & $+0.1\%$ \\
Utilization w.r.t.\ design torque $M_{\text{design}} = 1{,}740~\text{N}\cdot\text{m}$ & $54.3\%$ \\
Feasibility & No ($M_{t,\text{zul}} < M_{\text{design}}$) \\
\bottomrule
\end{tabular}
\end{table}

\subsubsection{Spline Test Case}
\label{subsubsec:test_spline}

The spline test case used overridden geometry parameters: minor diameter $d = 42~\text{mm}$, major diameter $D = 48~\text{mm}$, and tooth count $z = 8$. Table~\ref{tab:test_spline} presents the geometrical and material parameters, analytical results, and predicted torque capacity from the program. The model correctly computed the mean radius ($r_m = 22.5~\text{mm}$), projected flank height ($h_{\text{proj}} = 3.0~\text{mm}$), and allowable bearing pressure ($p_{\text{zul}} = 319.5~\text{MPa}$), all matching DIN~5480 ground truth values. The torque capacity of $6{,}470~\text{N}\cdot\text{m}$ ($6{,}470{,}000~\text{N}\cdot\text{mm}$) matches the ground truth value of $6{,}460~\text{N}\cdot\text{m}$ within $0.15\%$. The spline provides substantial capacity margin ($644\%$ above the required torque), demonstrating its suitability for high-torque applications.

\begin{table}[ht]
\centering
\caption{Spline test case: geometrical/material data, analytical results, and predicted torque. Ground truth values from~\cite{Result_SHC2_2024}.}
\label{tab:test_spline}
\begin{tabular}{lc}
\toprule
Parameter & Value \\
\midrule
\multicolumn{2}{l}{\textbf{Geometrical and Material Data}} \\
\midrule
Minor diameter $d$ & $42~\text{mm}$ \\
Major diameter $D$ & $48~\text{mm}$ \\
Tooth count $z$ & $8$ \\
Hub length $L$ & $50~\text{mm}$ \\
Shaft material & Steel E360 \\
Hub material & Steel 16MnCr5 \\
Required torque $M_{\text{req}}$ & $870~\text{N}\cdot\text{m}$ \\
Safety factor $S_R$ & $2.0$ \\
\midrule
\multicolumn{2}{l}{\textbf{Analytical Results}} \\
\midrule
Mean radius $r_m$ & $22.5~\text{mm}$ \\
Projected flank height $h_{\text{proj}}$ & $3.0~\text{mm}$ \\
Effective flank height $h_{\text{eff}}$ & $3.0~\text{mm}$ \\
Load distribution factor $K$ & $0.75$ \\
Allowable bearing pressure $p_{\text{zul}}$ & $319.5~\text{MPa}$ \\
\midrule
\multicolumn{2}{l}{\textbf{Predicted Torque from Program}} \\
\midrule
Torque capacity $M_{t,\text{zul}}$ & $6{,}470~\text{N}\cdot\text{m}$ \\
Ground truth (DIN~5480) & $6{,}460~\text{N}\cdot\text{m}$ \\
Deviation & $+0.15\%$ \\
Safety margin & $644\%$ \\
Feasibility & Yes \\
\bottomrule
\end{tabular}
\end{table}

These standardized test cases demonstrate that the analytical model achieves excellent agreement with DIN standard calculations, with discrepancies typically below $0.2\%$ for torque capacities and pressure values. The model correctly enforces both mechanical feasibility (torque capacity) and practical constraints (interference limits for press fits), validating its implementation against established engineering standards.

Figure~\ref{fig:torque_comparison} presents a comparison of the torque capacities from the analytical model relative to the design torque (required torque $\times$ safety factor = $870~\text{N}\cdot\text{m} \times 2.0 = 1{,}740~\text{N}\cdot\text{m}$), normalized to show the design torque as $100\%$. This visualization clearly demonstrates feasibility: connections with capacity above $100\%$ are feasible, while those below $100\%$ are infeasible. For the key connection, the torque capacity ($945~\text{N}\cdot\text{m}$) represents only $54.3\%$ of the design torque, confirming that the key is mechanically infeasible for this test case. In contrast, the spline connection provides a torque capacity ($6{,}470~\text{N}\cdot\text{m}$) representing $371.8\%$ of the design torque, indicating substantial capacity margin and clear feasibility. The press fit case is not included in this comparison as it was deemed infeasible due to interference limits, though the mechanical torque capacity ($3{,}951~\text{N}\cdot\text{m}$) was correctly calculated and would have been feasible ($227\%$ of design torque) if not for the manufacturability constraint.

\begin{figure}[H]
\centering
\includegraphics[width=0.7\textwidth]{figures/torque_comparison_chart.png}
\caption{Comparison of torque capacities from the analytical model relative to design torque (required torque $\times$ safety factor = $1{,}740~\text{N}\cdot\text{m}$, normalized to $100\%$). The key connection ($945~\text{N}\cdot\text{m}$, $54.3\%$) falls below the feasibility threshold, while the spline connection ($6{,}470~\text{N}\cdot\text{m}$, $371.8\%$) provides substantial capacity margin. Connections with capacity above $100\%$ are mechanically feasible. Source: own results.}
\label{fig:torque_comparison}
\end{figure}

For the same test configuration, the CatBoost classifier predicted \textbf{Spline} with $75.6\%$ confidence, assigning $19.1\%$ to Press fit and $5.2\%$ to Key. This aligns with analytical results: under the applied safety factor $S_R = 2.0$ and interference limits, only the spline remains mechanically feasible (press fit is rejected by the interference check and the key by insufficient torque capacity), with spline providing a $644\%$ capacity margin. The high confidence reflects the model's learned understanding that splines are preferred when they are the only connection type that fully satisfies both strength and manufacturability constraints for high-torque applications. Figure~\ref{fig:ml_test_result} shows the ML prediction output from the deployed web application.

\begin{figure}[H]
\centering
\includegraphics[width=0.6\textwidth]{figures/frontend_screenshots/ML_test_result.png}
\caption{ML model prediction output for the standardized test case configuration (shaft diameter $45~\text{mm}$, hub length $50~\text{mm}$, required torque $870~\text{N}\cdot\text{m}$, safety factor $2.0$, all preferences set to $0.5$). Source: own results.}
\label{fig:ml_test_result}
\end{figure}

% ---------------------------------------------------------------------------
\section{Synthetic Dataset Characteristics}
\label{sec:dataset_characteristics}

The synthetic dataset generation process produced 4{,}993 samples after filtering infeasible configurations~\cite{Picard_2023}. Diameters spanned $6$--$230~\text{mm}$ (mean: $55.3~\text{mm}$, std: $35.5~\text{mm}$), reflecting concentration in common engineering ranges (20--60~mm) while including both small and large diameter cases (Figure~\ref{fig:dataset_diameter}). Torque requirements ranged from $103~\text{N}\cdot\text{m}$ to $13.6~\text{MN}\cdot\text{m}$ (mean: $403~\text{kN}\cdot\text{m}$), covering both conservative and demanding applications (Figure~\ref{fig:dataset_torque}).

\begin{figure}[H]
\centering
\includegraphics[width=0.75\textwidth]{figures/dataset_diameter_hist.png}
\caption{Distribution of shaft diameters in the synthetic dataset. The histogram shows concentration in common engineering ranges (20--60~mm) while including both small and large diameter cases. Source: own results.}
\label{fig:dataset_diameter}
\end{figure}

\begin{figure}[H]
\centering
\includegraphics[width=0.75\textwidth]{figures/dataset_required_torque_hist_logx.png}
\caption{Distribution of required torque values in the synthetic dataset (log scale). The wide range from $10^2$ to $10^7$~N$\cdot$m reflects diverse application requirements. Source: own results.}
\label{fig:dataset_torque}
\end{figure}

The label distribution was: 54.7\% spline (2{,}729 samples), 28.6\% key (1{,}427 samples), and 16.8\% press fit (837 samples). This distribution reflects the analytical selector's behavior: splines often provide the highest capacity and are selected when torque demands are high or when preferences favor their advantages (durability, bidirectional capability, axial movement). Keys represent a cost-effective middle ground, while press fits are selected when they offer sufficient capacity with favorable preference alignment, particularly in scenarios with high vibration resistance or high-speed requirements. Figure~\ref{fig:dataset_class_dist} visualizes this class distribution.

\begin{figure}[H]
\centering
\includegraphics[width=0.75\textwidth]{figures/dataset_class_distribution.png}
\caption{Class distribution in the synthetic dataset. The imbalance reflects the analytical selector's behavior: splines are selected more frequently due to their high capacity and versatility, while press fits are selected in specific scenarios with favorable preference alignment. Source: own results.}
\label{fig:dataset_class_dist}
\end{figure}

The dataset included 3{,}964 samples with solid shafts and 1{,}029 with hollow shafts. For solid shafts, the inner diameter was set to zero (or not applicable), while for hollow shafts, the inner diameter had non-zero values ranging up to $128~\text{mm}$. Surface conditions were balanced: 50.8\% dry and 49.2\% oiled. Material distribution across 12 material types was approximately uniform, with each material appearing in roughly 300--360 samples, as shown in Figure~\ref{fig:dataset_material}. Safety factors ranged from 1.0 to 2.0, with a mean of 1.59 and standard deviation of 0.16, reflecting realistic engineering practice. The safety factor distribution is illustrated in Figure~\ref{fig:dataset_safety_factor}.

\begin{figure}[H]
\centering
\includegraphics[width=0.85\textwidth]{figures/dataset_material_distribution_top15.png}
\caption{Distribution of the top 15 most common materials in the synthetic dataset. The approximately uniform distribution across materials ensures balanced representation for ML training. Source: own results.}
\label{fig:dataset_material}
\end{figure}

\begin{figure}[H]
\centering
\includegraphics[width=0.75\textwidth]{figures/dataset_safety_factor_hist.png}
\caption{Distribution of safety factors in the synthetic dataset. The mean value of 1.59 reflects realistic engineering practice, with adjustments based on bending loads, surface conditions, and user preferences. Source: own results.}
\label{fig:dataset_safety_factor}
\end{figure}

The dataset included many ``boundary'' cases where two connection types were feasible, enabling the ML model to learn subtle decision boundaries. For cases with diameters in the range of $20$--$40~\text{mm}$ and moderate torque, the label varied (press fit, key, or spline) depending on preferences and exact torque, indicating realistic trade-offs rather than trivial diameter-to-label mapping. Preference analysis confirmed that labels switched when preferences crossed thresholds, validating that the preference scoring effectively influenced outcomes. Overall, the synthetic dataset was diverse and representative of the intended design space, supporting robust ML training.

% ---------------------------------------------------------------------------
\section{Requirements Validation}
\label{sec:requirements_validation}

This section maps test results to the requirements established in the methodology chapter (Chapter~\ref{ch:methodology}), demonstrating that the developed system meets its design objectives.

\subsection{Requirement R1: Scoring System Accuracy}
\label{subsec:req_scoring}

\textbf{Requirement:} The analytical scoring system must correctly identify feasible connections and rank them according to mechanical capacity and user preferences.

\textbf{Test:} The analytical model was verified against known engineering cases (Section~\ref{sec:analytical_verification}), including standard shaft--hub configurations with documented torque capacities.

\textbf{Result:} The analytical model correctly identified feasible connections and computed torque capacities within a few percent of DIN standard calculations (DIN~7190, DIN~6885, DIN~5480). For the standardized test case (45~mm shaft, 870~N$\cdot$m required torque, safety factor 2.0), the model correctly identified that only the spline connection is mechanically feasible: the key connection fails due to insufficient capacity (54.3\% of design torque), and the press fit fails the interference manufacturability check despite having adequate torque capacity. This matches engineering judgment and validates the model's ability to enforce both mechanical and practical constraints.

\textbf{Status:} \textbf{Pass}: The scoring system demonstrates accuracy consistent with standard design practice.

\subsection{Requirement R2: Dataset Diversity and Coverage}
\label{subsec:req_dataset}

\textbf{Requirement:} The synthetic dataset must cover a broad spectrum of realistic engineering scenarios, including diverse diameters, torque requirements, and material combinations.

\textbf{Test:} Dataset characteristics were analyzed (Section~\ref{sec:dataset_characteristics}), including diameter distribution (6--230~mm), torque range (103~N$\cdot$m to 13.6~MN$\cdot$m), and material coverage (12 material types).

\textbf{Result:} The dataset contains 4{,}993 samples with balanced representation across diameter ranges, torque levels, and material combinations. Boundary cases where multiple connection types are feasible are well-represented, enabling the ML model to learn subtle decision boundaries.

\textbf{Status:} \textbf{Pass}: The dataset demonstrates sufficient diversity and coverage for robust ML training.

\subsection{Requirement R3: ML Model Performance}
\label{subsec:req_ml_performance}

\textbf{Requirement:} The ML classifier must achieve balanced performance across all three connection classes, with macro F1-score exceeding 0.75.

\textbf{Test:} Multiple models were trained and evaluated on a test set with groud-truth (Section~\ref{sec:ml_performance}), using macro-averaged F1-score as the primary metric.

\textbf{Result:} CatBoost achieved a macro F1-score of 0.7986, exceeding the requirement. Per-class performance: press fit F1=0.6774, key F1=0.8057, spline F1=0.9127.

\textbf{Status:} \textbf{Pass}: ML model performance meets and exceeds the specified requirement.

\subsection{Requirement R4: System Integration and Usability}
\label{subsec:req_integration}

\textbf{Requirement:} The system must integrate analytical and ML components into a unified web application providing real-time recommendations with transparent outputs.

\textbf{Test:} The web application was tested with multiple use-case scenarios (Section~\ref{sec:webapp_demo}), evaluating both analytical and ML outputs.

\textbf{Result:} The integrated system successfully provides side-by-side analytical and ML recommendations, displaying torque capacities, feasibility status, and confidence scores. User testing confirmed the interface is intuitive and the outputs are interpretable.

\textbf{Status:} \textbf{Pass}: System integration meets usability and transparency requirements.

% ---------------------------------------------------------------------------
\section{Machine Learning Model Performance}
\label{sec:ml_performance}

The ML models were trained on 3{,}994 samples (80\% of the dataset) and evaluated on a held-out test set of 999 samples (20\%). All models were configured with 150 estimators and evaluated using macro-averaged F1-score as the primary selection criterion to ensure balanced performance across all connection classes. Table~\ref{tab:model_comparison} summarizes the performance metrics for all evaluated models.

\begin{table}[ht]
\centering
\caption{Model performance comparison on test set (999 samples)}
\label{tab:model_comparison}
\begin{tabular}{lcccc}
\toprule
Model & Accuracy & Precision (macro) & Recall (macro) & F1-score (macro) \\
\midrule
Random Forest & 0.7788 & 0.7513 & 0.6813 & 0.7031 \\
XGBoost & 0.8278 & 0.7931 & 0.7702 & 0.7802 \\
LightGBM & 0.8198 & 0.7815 & 0.7593 & 0.7686 \\
CatBoost & \textbf{0.8458} & \textbf{0.8125} & \textbf{0.7879} & \textbf{0.7986} \\
Ensemble & 0.8268 & 0.7942 & 0.7627 & 0.7759 \\
\bottomrule
\end{tabular}
\end{table}

CatBoost achieved the highest macro F1-score of 0.7986 and was selected as the best model. Its superior performance is attributed to its built-in handling of categorical features (shaft type, material, surface condition) without requiring explicit one-hot encoding, thereby preserving categorical relationships more effectively than other models. The ensemble model, which combines all four base models through soft voting, achieved slightly lower performance (F1-macro: 0.7759) than CatBoost alone, suggesting that CatBoost's predictions were already well-calibrated and the ensemble did not provide additional benefit in this case.

Table~\ref{tab:per_class_metrics} presents per-class precision, recall, and F1-scores for the selected CatBoost model. The model performs well across all three classes, with spline showing the highest performance (F1: 0.9127), followed by key (F1: 0.8057) and press fit (F1: 0.6774). The lower performance on press fits reflects the class imbalance in the dataset (press fits represent only 16.8\% of samples) and the fact that press fits are often selected in boundary cases where multiple connection types are feasible, making them harder to predict accurately.

To understand which features most influence the model's predictions, feature importance analysis was performed on the CatBoost model. Figure~\ref{fig:feature_importance} shows the top 20 most important features. The analysis reveals that geometric parameters (shaft diameter, required torque, hub length) and preference weights (particularly durability, cost, and maintenance) are the primary drivers of connection type selection. This aligns with engineering intuition: diameter and torque directly determine mechanical feasibility, while preferences differentiate between multiple feasible options. The importance of preference weights demonstrates that the model successfully learned to incorporate user priorities into its predictions, validating the hybrid approach's design objective.

\begin{figure}[H]
\centering
\includegraphics[width=0.9\textwidth]{figures/feature_importance_CatBoost copy.png}
\caption{Top 20 feature importances for the CatBoost model. Geometric parameters (shaft diameter, required torque) and preference weights (durability, cost, maintenance) are the primary drivers of connection type selection, validating the hybrid approach's design. Source: own results.}
\label{fig:feature_importance}
\end{figure}

\begin{table}[ht]
\centering
\caption{Per-class metrics for CatBoost model}
\label{tab:per_class_metrics}
\begin{tabular}{lccc}
\toprule
Class & Precision & Recall & F1-score \\
\midrule
Key & 0.8143 & 0.7972 & 0.8057 \\
Press & 0.7343 & 0.6287 & 0.6774 \\
Spline & 0.8889 & 0.9377 & 0.9127 \\
\bottomrule
\end{tabular}
\end{table}

The confusion matrix for CatBoost (Figure~\ref{fig:confusion_matrix_catboost}) reveals the primary sources of misclassification. Most errors occur between press fit and key (24 key samples misclassified as press, 32 press samples misclassified as key), and between key and spline (34 key samples misclassified as spline). These confusions align with engineering intuition: press fits and keys are both common for moderate torque applications, while keys and splines share similar form-closure characteristics. The model rarely confuses press fits with splines (20 press misclassified as spline, 14 spline misclassified as press), reflecting their fundamentally different torque transmission mechanisms.

\subsection{Error Analysis and Failure Modes}
\label{subsec:error_analysis}

Analysis of misclassifications reveals systematic patterns: the largest source of error (56 errors, 36.4\% of total misclassifications) occurs between press fits and keys (24 key samples misclassified as press, 32 press samples misclassified as key), primarily in moderate torque scenarios (200--2000~N$\cdot$m) with diameters between 20--50~mm, where both connection types are mechanically feasible. This reflects the inherent ambiguity in boundary regions where multiple solutions are equally valid. Key--spline confusion (54 errors total: 34 key samples misclassified as spline, 20 spline samples misclassified as key, representing 35.1\% of errors) typically occurs in higher-torque scenarios where keys approach capacity limits; this conservative over-design is less problematic than under-design. Press fit--spline confusion (44 errors total: 30 press samples misclassified as spline, 14 spline samples misclassified as press, representing 28.6\% of errors) is least frequent, occurring mainly with extreme parameter combinations. The lower F1-score for press fits (0.6774) is attributed to class imbalance (16.8\% of samples) and boundary case complexity. The hybrid approach mitigates these limitations by presenting analytical results alongside ML predictions, allowing users to make informed decisions when multiple options are feasible.

\begin{figure}[H]
\centering
\includegraphics[width=0.7\textwidth]{figures/confusion_matrix_CatBoost.png}
\caption{Confusion matrix for CatBoost model (rows: true class, columns: predicted class). The diagonal elements represent correct predictions, while off-diagonal elements show misclassifications. Source: own results.}
\label{fig:confusion_matrix_catboost}
\end{figure}

\begin{figure}[H]
\centering
\includegraphics[width=0.7\textwidth]{figures/confusion_matrix_Random_Forest.png}
\caption{Confusion matrix for Random Forest model. Source: own results.}
\label{fig:confusion_matrix_rf}
\end{figure}

\begin{figure}[H]
\centering
\includegraphics[width=0.7\textwidth]{figures/confusion_matrix_XGBoost.png}
\caption{Confusion matrix for XGBoost model. Source: own results.}
\label{fig:confusion_matrix_xgb}
\end{figure}

\begin{figure}[H]
\centering
\includegraphics[width=0.7\textwidth]{figures/confusion_matrix_LightGBM.png}
\caption{Confusion matrix for LightGBM model. Source: own results.}
\label{fig:confusion_matrix_lgb}
\end{figure}

\begin{figure}[H]
\centering
\includegraphics[width=0.7\textwidth]{figures/confusion_matrix_Ensemble.png}
\caption{Confusion matrix for Ensemble model (soft-voting combination of all four base models). Source: own results.}
\label{fig:confusion_matrix_ensemble}
\end{figure}

Training times were measured for each model: Random Forest completed in 0.40 seconds, XGBoost in 2.25 seconds, LightGBM in 2.44 seconds, and CatBoost in 1.19 seconds. The ensemble required 5.44 seconds to train all four base models. Prediction times were all under 50 milliseconds, with CatBoost requiring 31 milliseconds per prediction. This computational efficiency makes the ML model suitable for real-time decision support in the web application.


% ---------------------------------------------------------------------------
\section{Web Application Demonstration}
\label{sec:webapp_demo}


The system provides rich output: not just a single answer, but context showing how close each option is to its limits and how confidence is distributed. Side-by-side comparison of analytical versus ML recommendations increases trust, and visual torque bars help quickly understand performance. Preference sliders allow quick exploration, enabling users to simulate expert thought processes by adjusting sliders to see when recommendations switch.

\begin{figure}[H]
    \centering
    \includegraphics[width=0.9\textwidth]{figures/frontend_screenshots/frontend_basic1.png}
    \caption{Overview of the web application frontend. The landing page guides the user through entering geometry, operating conditions, and preferences before requesting recommendations. Source: Screenshot of the developed web application.}
    \label{fig:webapp_frontend_overview}
\end{figure}

\begin{figure}[H]
    \centering
    \includegraphics[width=0.9\textwidth]{figures/frontend_screenshots/frontend_params1.png}
    \caption{Input form for the standardized test case, showing shaft geometry, torque, safety factor, materials, and surface condition fields. These parameters correspond to the analytical verification setup described in Section~\ref{sec:analytical_verification}. Source: Screenshot of the developed web application.}
    \label{fig:webapp_frontend_params}
\end{figure}

\begin{figure}[H]
    \centering
    \includegraphics[width=0.9\textwidth]{figures/frontend_screenshots/frontend_prediction1.png}
    \caption{Frontend view of the recommendation output. The interface presents analytical feasibility and torque utilization for all three connection types. The ML class probabilities are shown earlier with the test results in Figure \ref{fig:ml_test_result}. Source: Screenshot of the developed web application.}
    \label{fig:webapp_frontend_prediction}
\end{figure}

