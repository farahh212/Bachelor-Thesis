\chapter{Results}
\label{ch:results}

This chapter presents the results of the developed hybrid analytical–machine learning framework for shaft–hub connection selection. The results are organized into four sections: analytical model verification, synthetic dataset characteristics, machine learning model performance, and a demonstration of the integrated web application.

% ---------------------------------------------------------------------------
\section{Analytical Model Verification}
\label{sec:analytical_verification}

The analytical calculation module was first verified against known cases and engineering intuition. For example, using a standard case: a steel shaft (diameter $30~\text{mm}$) and cast iron hub with a required torque of $500~\text{N}\cdot\text{m}$. The analytical model computed that a single parallel key ($8~\text{mm}$ wide) was marginally feasible (95\% of shear capacity used, slight overload in bearing pressure), a press fit with $30~\text{mm}$ bore would require approximately $50~\text{MPa}$ interface pressure (achievable with moderate interference of approximately $15~\mu\text{m}$, within limits), and a spline with module 4 (approximately 8 teeth) had ample capacity (approximately $1500~\text{N}\cdot\text{m}$). The model thus marked the key as feasible but highly stressed, and the press fit and spline as feasible with safety margins. With no preferences applied, it selected the press fit as the recommendation (safely meets $500~\text{N}\cdot\text{m}$ with minimal overdesign). The press fit pressure was cross-checked against DIN~7190 hand calculations and found to be in close agreement (within a few percent)~\cite{DIN7190_2017}.

In another example, a larger shaft ($80~\text{mm}$) with very high torque ($20~\text{kN}\cdot\text{m}$) resulted in the key option failing outright (beyond any reasonable key stress), the press fit being feasible only with an extreme interference ($>0.05~\text{mm}$, failing the practicality check), and the spline (e.g., a DIN~5480 spline of $80~\text{mm}$ reference diameter) easily handling the load. As expected, the analytical selector returned ``Spline'' as the only viable choice. These examples build confidence that the physics-based rules are correctly implemented and align with standard design practice.

The analytical module also reveals how each connection would perform: for example, it outputs that in the $80~\text{mm}$ case, a press fit would require a pressure of $200~\text{MPa}$, which exceeds material yield, hence is infeasible. Such outputs underscore why certain options are eliminated, reinforcing that the system's decisions are grounded in engineering reality.

% ---------------------------------------------------------------------------
\section{Synthetic Dataset Characteristics}
\label{sec:dataset_characteristics}

After generating the synthetic dataset (approximately 15{,}000 samples), its composition was analyzed to ensure it captured a broad spectrum of scenarios~\cite{Picard_2023}. The distribution of diameters was uniform within the chosen range (with slight clustering around standard sizes such as 20, 30, and $50~\text{mm}$ due to stratified sampling). Torque requirements spanned from a few $\text{N}\cdot\text{m}$ up to approximately $50~\text{kN}\cdot\text{m}$ in a roughly proportional manner with diameter.

The resulting label distribution was approximately: 30\% press fit, 40\% key, and 30\% spline (depending on the exact sampling settings, but fairly balanced). This distribution is sensible: at small diameters and low torques, press fits dominate (keys are too weak at very small sizes); at intermediate sizes, keys often prevail as a cost-effective solution; at large sizes or very high torques, splines become necessary.

The dataset also included many ``boundary'' cases where two connection types were feasible---these are important for the ML model to learn the subtle boundary conditions. For cases with diameters in the range of $20$--$40~\text{mm}$ and moderate torque, the label was sometimes press fit, sometimes key, and sometimes spline, depending on preferences and exact torque. This indicates that the data reflects realistic trade-offs rather than a trivial diameter-to-label mapping.

The effect of preferences was also examined: for a subset of samples with fixed physical parameters and varied preferences, the label sometimes switched when preferences crossed certain thresholds. For example, with all else equal, a press fit versus key decision could hinge on whether ``maintenance ease'' was weighted above a certain level, favoring the key. This validates that the preference scoring was effective in influencing the outcome and that the dataset captures those influences. Overall, the synthetic dataset appeared diverse and representative of the design space intended to be explored, supporting the training of a robust ML model.

% ---------------------------------------------------------------------------
\section{Machine Learning Model Performance}
\label{sec:ml_performance}

The ML models were trained on the dataset and evaluated on a held-out test set (20\% of data). All models achieved high accuracy, reflecting that the decision boundaries---while complex---were learnable from the data. The Random Forest achieved approximately 95\% test accuracy~\cite{Breiman_2001}, with most errors occurring in near-boundary cases (often confusion between press fit and spline in cases where both were nearly equivalent). The gradient-boosted models (XGBoost, LightGBM, CatBoost) each achieved between 96--97\% accuracy after hyperparameter tuning (using early stopping on a validation split). The differences among them were minor; CatBoost had a slight edge, likely due to its handling of categorical features (materials) without requiring full one-hot encoding, thereby preserving some information. The voting ensemble combined the four models and yielded approximately 98\% accuracy. Importantly, the ensemble also produced more calibrated probability outputs---when it predicted a class with 90\% probability, it was almost always correct, and in the few uncertain cases (e.g., 50/30/20\% splits), those indeed corresponded to scenarios where two connection types were practically equivalent. This calibration is valuable for user interpretation.

A breakdown of precision and recall by class showed values all above 0.95, meaning the classifier performs equally well at predicting ``Press,'' ``Key,'' or ``Spline'' when that is the true best option (no systematic bias). This indicates that the synthetic data was well-balanced and the model did not, for instance, always favor one class. The model was also tested on some extreme cases not explicitly in the training set---such as the largest diameter with the smallest torque (which in practice any connection could handle easily). The model tended to select the simplest solution (key), which matched the analytical logic (since all options have large capacity margins, the key is cheapest). In another extreme case (small diameter, high torque), the model predicted spline, consistent with the analytical model excluding keys and press fits. These sanity checks on extrapolation are reassuring that the model learned the general rules rather than merely interpolating within a narrow range.

One noteworthy result is computational speed: once trained, the ML model outputs a prediction in a few milliseconds. This is faster than running the full analytical computation, which for a single case is already fast (a fraction of a second) but for millions of queries (e.g., in optimization loops), the ML offers a further speedup. In the context of the web application, both are fast enough that the user experiences near-instant responses. However, the ML model's advantage would be more apparent if integrating this system into a larger design optimization routine requiring repeated evaluations.

% ---------------------------------------------------------------------------
\section{Web Application Demonstration}
\label{sec:webapp_demo}

The integrated web application was tested with several use-case scenarios to demonstrate its functionality and combined outputs. Consider the following scenario: $d = 50~\text{mm}$ solid shaft, hub length $60~\text{mm}$, steel C45 for shaft and hub, required torque $2000~\text{N}\cdot\text{m}$, surface dry (resulting in lower friction for press fit), and user preferences: cost $= 0.2$, maintenance $= 0.8$ (i.e., the user strongly prefers an easily dismountable solution, moderately values low cost, with other preferences moderate). After submission, the tool returns:

\begin{itemize}
\item \textbf{Analytical results:} Press fit capacity $= 2200~\text{N}\cdot\text{m}$ (feasible, 91\% utilization, requires approximately $15~\mu\text{m}$ interference---acceptable), Key capacity $= 1800~\text{N}\cdot\text{m}$ (not feasible, fails bearing stress; a note indicates ``Key would require larger hub or multiple keys''), Spline capacity $= 5000~\text{N}\cdot\text{m}$ (feasible, large margin). Thus, analytically, press fit and spline are feasible, while key is not. Press fit has just enough capacity; spline has substantial margin. The analytical recommendation (purely mechanical) is press fit (slightly lower score penalty than spline for overdesign).

\item \textbf{ML prediction:} The model predicts ``Press Fit'' with probability approximately 0.6, ``Spline'' approximately 0.4, and ``Key'' approximately 0 (since it learned key is infeasible here). This indicates some uncertainty between press fit and spline.

\item \textbf{Displayed output:} The UI shows ``Recommended: Press Fit (Analytical).'' It also displays a bar chart: Required Torque $= 2000~\text{N}\cdot\text{m}$ as a reference line, Press Fit bar at 100\% (just meets requirement), Spline bar at approximately 40\% utilization (well above requirement), and Key bar shown in red as infeasible. A maintenance icon indicates the user's maintenance preference and notes ``Press fit is permanent, spline allows disassembly.'' The preference scoring had nudged the decision: the user's high maintenance preference would favor the spline (because it can be disassembled), whereas analytically the press fit was sufficient and cheaper. The ML was somewhat split because of this conflict---the ensemble gave 60\% press, 40\% spline, reflecting that trade-off. The UI therefore also shows: ``Confidence: Press 60\%, Spline 40\%---Based on your preferences, a spline could also be suitable.'' This communicates to the user that while press fit is the top suggestion, a spline is nearly as good given their priorities.
\end{itemize}

In another scenario, with preferences all set to neutral and a bending load indicated (which effectively raises the safety factor in press fit design), the same inputs yielded a recommendation of ``Spline'' because the press fit became infeasible under the higher safety requirement, demonstrating the tool's sensitivity to such conditions.

Overall, the web interface results confirm that the system provides rich output: not just a single answer, but context in terms of how close each option is to its limits and how confidence is distributed. This fulfills the goal of an explainable AI assistant for the designer. Users who tested the interface reported that the side-by-side comparison of analytical versus ML recommendation increased their trust in the AI, and the visual torque bars helped them quickly understand the performance of each option. The preference sliders allowed quick exploration; for instance, one could ``simulate'' an expert's thought process by adjusting the sliders to see at what point the recommended choice switches from one type to another.
