\chapter*{Abstract}
\addcontentsline{toc}{chapter}{Abstract}

Shaft–hub connections are essential components used to transmit torque between mechanical elements. A wide variety of connection types exist, and using traditional methods such as engineering judgment and handbook equations, a suitable option can be selected. This thesis develops a model capable of identifying the most appropriate shaft–hub connection among three common alternatives: interference (press) fits, keyed fits, and splined fits, thereby automating the selection process.

A key motivation for this work arises from a practical gap: no publicly available dataset exists for shaft–hub connection selection, making machine-learning approaches difficult to pursue directly. To address this, the first part of the thesis focuses on constructing a synthetic dataset using analytical equations together with a preference-based scoring mechanism. The scoring system incorporates eight user-defined preference dimensions: assembly/disassembly ease, axial movement suitability, manufacturing cost, bidirectional torque capability, vibration resistance, high-speed suitability, maintenance ease, and durability/fatigue life. A pipeline was developed to assign a shaft–hub connection to individual scenarios until the scoring behaviour and feasibility logic achieved satisfactory performance. Once validated, this pipeline was employed as an automated label generator to produce a large and diverse dataset that reflects realistic engineering designs, with input parameters randomized in accordance with DIN standards.

The second part of the thesis develops a classification model trained on this synthetic dataset. The model learns to predict the most suitable connection type based on geometric parameters, material combinations, surface conditions, torque requirements, and user preferences. By integrating analytical constraints with learned behaviour, the resulting classifier captures both practical feasibility relations and subtler preference-driven trade-offs.

To support accessibility and real-world usage, a web-based interface was implemented using FastAPI and React. The interface utilizes the trained machine-learning model for recommendations, presenting the predicted connection type together with its softmax classification score for each query. In addition, analytical torque capacities for all connection types are computed and displayed, providing users with a transparent and interpretable comparison between ML confidence and practical feasibility. Users can freely specify their design parameters and preference values, making the system flexible for educational and practical applications.

The resulting system offers an explainable, data-driven, and mechanically consistent platform. It demonstrates how analytical engineering knowledge can be transformed into a synthetic dataset for training intelligent models, ultimately improving the accessibility of shaft–hub design expertise and helping users better understand the trade-offs between different connection types.


